\chapter{Pedestal Modeling and Theory}\label{ch:Modeling}

While a number of high-performance regimes (described in \cref{ch:HighPerformance}) have been established and are actively explored for tokamak operation, much of the physics governing these regimes is still unknown.  In particular, the physics underlying the structure of the pedestal is an area of active research, due in large part to the inherent difficulty in experimentally diagnosing the pedestal plasma as it varies over short scale lengths, and in the wide variability of H-mode behaviors observed in tokamak experiments.  Nevertheless confidence in the prediction of pedestal height and stability for ITER- and reactor-scale devices is essential: core temperature and pressure in the plasma are strongly sensitive to pedestal conditions due to profile stiffness driven by marginally-stable temperature-gradient modes \cite{Hubbard1998}\gnote{elaborate or reword}, thus fusion power density is controlled by the pressure pedestal structure.  Moreover, operation with large, uncontrolled Edge-Localized Modes (ELMs -- see \cref{sec:hcr_elmy}) can drive transient heat loads exceeding wall material tolerances on ITER-scale devices \cite{Loarte2003,Federici2003} -- an understanding of pedestal stability against ELMs is necessary for ITER operation and beyond.  This chapter provides a review of the efforts to date in theory and modeling of the pedestal, including the theoretical models used in the balance of this thesis.\nicesectionending\gnote{elaborate?}

\section{Early Models}\label{sec:mod_early}

\gnote{needs better title...}Initial efforts in understanding the pedestal took a variety of approaches, including models built from fairly simple \emph{ans\:atze} for the physics determining the pedestal structure.  Several of these approaches are detailed here.  Overviews of these models may also be found in \cite[\S 2]{Hubbard2000,Hughes2005}.

\subsection{Neutral Penetration Models}\label{subsec:mod_neutral}

\subsection{Ion-Orbit Loss Models}\label{subsec:mod_ionorbitloss}

Due to the importance in the edge $E_r$ well in pedestal formation, modeling efforts naturally turned to potential sources for the electric field to explain the pedestal.  One suggested source was ion orbit loss across the last closed flux surface, in which the gyro-motion of ions near the edge intersect the SOL or the plasma-facing material surfaces -- the charge imbalance induced by this particle ``leak'' results in a radial electric field \cite{Shaing1990}.  Assuming ion orbit losses drive the $E_r$ well, the $\vec{E}\times\vec{B}$ shear layer width ought to be governed by the banana orbit width, which scales as the poloidal gyroradius $\rho_{i,pol} \sim \sqrt{T_i}/B_p$.  Accounting for the squeezing effect of the radial electric field on the banana orbit width, Shaing \cite{Shaing1992} gives for the well width

\begin{equation}\label{eq:Shaing_width}
 \begin{aligned}
  &\Delta_{\vec{E}\times\vec{B}} \propto \sqrt{\varepsilon} \frac{\rho_{i,pol}}{\sqrt{S}}\\
  &S = \left| 1 - \frac{1}{B_p \omega_{ci,p}} \frac{dE_r}{dr}\right|
 \end{aligned}
\end{equation}

\noindent where $S$ is the squeezing factor and $\omega_{ci,p}$ is the ion cyclotron frequency evaluated with the poloidal field.  The model is further refined by Itoh \& Itoh \cite{Itoh1996} to include the broadening effects of viscosity shear.

\subsection{Empirical Observations}\label{subsec:mod_empirical}

\nicesectionending

\section{MHD Stability: Peeling-Ballooning Modes}\label{sec:mod_pb}

\subsection{Ballooning MHD}\label{subsec:mod_balloon}

\subsection{Peeling MHD}\label{subsec:mod_peel}

\subsection{ELITE Code}\label{subsec:mod_elite}

\nicesectionending

\section{Turbulent Modeling}\label{sec:mod_turbulence}

\nicesectionending

\section{The EPED Model}\label{sec:mod_eped}

\nicechapterending

\bibliographystyle{../plainurl}
\bibliography{../references}