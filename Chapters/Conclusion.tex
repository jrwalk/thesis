\chapter{Conclusions \& Future Work}\label{ch:Conclusion}

The work described in this thesis has contributed to the study of the pedestal in high-performance regimes -- as the pedestal sets a strict constraint on the core pressure and fusion power density, as well as deterimining the stability against large, deleterious Edge-Localized Modes (ELMs), a firm understanding of the physics entailed in the structure and stability of the pedestal is essential to the development of operating scenarios for ITER- and reactor-scale devices.  To this end, this thesis details a combined approach using both empirical observations and theoretical/computational models to understand the governing physics of the pedestal.\gnote{opener section should be work predating thesis -- be sure this is clear}

These analysis methods are developed first for ELMy H-mode experiments on Alcator C-Mod -- as this is the most broadly-accessible and well-understood high-performance regime on most tokamak devices (see \cref{ch:HighPerformance}), ELMy H-mode is considered the baseline for high-confinement operation on ITER \cite{Shimada2007}.  However, large, uncontrolled ELMs can drive pulsed heat loading and erosion damage in excess of the material tolerances of plasma-facing components.  Understanding the limits placed on the pedestal by ELM stability, then, is of critical importance for planned operation on ITER.  The work presented in this thesis (see also \cite{Walk2012}) was undertaken as a contribution to a joint research effort across several machines \cite{Groebner2013} to develop a predictive model for the ELMy H-mode pedestal.  

The empirical and computational analyses developed for ELMy H-mode are also applied in this thesis to the I-mode \cite{Whyte2010}, a novel high-confinement regime developed on Alcator C-Mod, with a number of highly desirable characteristics for reactor operation.  I-mode is notable in that it decouples energy and particle transport, forming an H-mode-like temperature pedestal with high energy confinement while maintaining an L-mode-like density profile with low particle confinement and favorably rapid transport of impurities from the plasma.  Moreover, the energy confinement in I-mode appears to degrade significantly more weakly than H-mode.  The work in this thesis (see also \cite{Walk2014}) focuses on characterizing the I-mode pedestal in terms of its structure and impact on global performance -- and therefore implications for operation on larger devices -- and its inherent avoidance of the instabilities associated with the ELM trigger.\nicesectionending

\section{Contributions to ELMy H-Mode Physics}\label{sec:conc_elmy}

ELMy H-mode experiments on C-Mod, described in \cref{ch:Elmy}, significantly expand the parameter range of investigation: C-Mod operation reaches the highest thermal pressure of any tokamak, up to within a factor of $\sim 2$ of the target pedestal pressure for ITER, as well as reaching the highest magnetic field at $\SI{8}{\tesla}$.  These experiments also entailed a significant scan in plasma current ($400-\SI{1100}{\kilo\ampere}$) and field ($3.5-\SI{8}{\tesla}$), representing a broad parameter range well outside of the explored operational space on other devices, at ITER-like conditions in several respects (\eg field, density, pressure).

%The pedestal in ELMy H-mode\gnote{move this up a section, or shorten to reference?} is thought to be limited by two key physics aspects: first, that the rapid onset of peeling-ballooning MHD instabilities (described in \cref{sec:mod_pb}) driven by the steep pressure gradients and high bootstrap current densities in the pedestal drives the explosive burst of ELM transport in the edge, and second, that the strong transport driven by kinetic-ballooning mode (KBM) turbulence, described in \cref{sec:mod_turbulence}, above an onset threshold limits the pedestal (particularly the pressure gradient) leading up to the ELM.  

The observed behavior in the pedestal are consistent with limits based in peeling-ballooning MHD instability and kinetic-ballooning turbulence (described in \cref{sec:mod_pb} and \cref{sec:mod_turbulence}, respectively).  Over the scan in plasma current (see \cref{sec:elmy_engineer}), the pressure pedestal height exhibits a trend of $p_{ped} \sim I_p$, while the pressure pedestal width scales as $\Delta_{p_e} \sim I_p^{-1}$.  Within this trend, the density and temperature pedestal widths individually exhibit no systematic trend with the plasma current, while the density and temperature pedestal heights both exhibit a weakly-positive trend with current.  This behavior is consistent with the expected $\nabla p \sim I_p^2$ constraint expected from the ballooning MHD stability limit; however, there is significant scatter, which may be corrected with a more refined analysis accounting for the pedestal width.  The attainable poloidal beta at the pedestal top is set largely by plasma shaping -- this is consistent with the ballooning MHD limit, constrained in terms of $\alpha_{MHD} \sim d\beta_{p}/d\psi$, which over the fairly restricted pedestal width sets (to zero'th order approximation) a limit on $\beta_{p,ped}$, with the attainable beta increasing with shaping due its stabilizing effect on the MHD mode.

The pedestal width is seen to scale as $\Delta \sim \beta_{p,ped}^{1/2}$, as expected from KBM turbulence (\cf \cref{subsec:elmy_eped_width}, particularly \cref{fig:elmy_betap_deltapsi_betabin}).  Alternate width models (\cref{subsec:elmy_width_oldmodels}) are largely discounted -- the density pedestal width does not exhibit the inverse dependence on the pedestal density expected from neutral-penetration models, while the temperature pedestal exhibits no dependence on the poloidal gyroradius, in contrast to predictions based on ion-orbit loss influence on the $\vec{E}\times\vec{B}$ sheared flow.  The pressure pedestal width variation with poloidal gyroradius is explained by the strong co-variance between $\rho_{i,pol}$ and $\beta_{p,ped}$.  The scatter in the pedestal height versus plasma current is corrected by accounting for this variation in width.  Taking the pedestal height to scale as $p_{ped} \sim \nabla p \times \Delta_p$, the pedestal height is predicted well by the combination of the ballooning MHD limit and the width scaling above, $p_{ped} \sim I_p^2 \beta_{p,ped}^{1/2} \sim I_p \sqrt{n_{e,ped} T_{e,ped}}$.

Strictly, the KBM model allows for secondary variation on shaping and collisionality on the mode threshold, taking the form $\Delta = G(\nu^*,\varepsilon,...) \beta_{p,ped}^{1/2}$ where $G$ is a weakly-varying function (taken to be constant in the simplest model).  The scale function is fitted to an average value $\langle G \rangle = 0.0857$, with no systematic depedence of the width on field, collisionality, shaping, or toroidal gyroradius observed (\cref{subsec:elmy_normwidth}).  

ELMy H-mode pedestals are also tested against results from the EPED model \cite{Snyder2011}, described in \cref{sec:mod_eped}, which operates based on self-consistent calculations of the peeling-ballooning MHD and KBM turbulence limits.  Notably, EPED bases its calculations only on engineering target parameters, such that it may be used predictively, rather than relying on reconstructed experimental data.  EPED predictions (see \cref{sec:elmy_eped}) for the pressure pedestal width and height match the observed results to within the $\sim 20\%$ systematic uncertainty of the model, with the best correspondence between prediction and experiment when the observed data is treated to only use time frames immediately preceding the ELM crash, when the pedestal structure most closely matches the point of instability.\nicesectionending

\section{Contributions to I-Mode Physics}\label{sec:conc_imode}

\subsection{Empirical Observations}\label{subsec:conc_imode_emp}

I-mode exhibits a number of highly-desirable behaviors for reactor-scale operation:

\subsection{Stability Modeling}\label{subsec:conc_imode_mod}

\nicesectionending

\section{Future Work}\label{sec:conc_future}

\nicechapterending

\bibliographystyle{../plainurl}
\bibliography{../references}