\chapter{Conclusions \& Future Work}\label{ch:Conclusion}

The work described in this thesis has contributed to the study of the pedestal in high-performance regimes -- as the pedestal sets a strict constraint on the core pressure and fusion power density, as well as deterimining the stability against large, deleterious Edge-Localized Modes (ELMs), a firm understanding of the physics entailed in the structure and stability of the pedestal is essential to the development of operating scenarios for ITER- and reactor-scale devices.  To this end, this thesis details a combined approach using both empirical observations and theoretical/computational models to understand the governing physics of the pedestal.

These analysis methods are developed first for ELMy H-mode experiments on Alcator C-Mod -- as this is the most broadly-accessible and well-understood high-performance regime on most tokamak devices (see \cref{ch:HighPerformance}), ELMy H-mode is considered the baseline for high-confinement operation on ITER \cite{Shimada2007}.  However, large, uncontrolled ELMs can drive pulsed heat loading and erosion damage in excess of the material tolerances of plasma-facing components.  Understanding the limits placed on the pedestal by ELM stability, then, is of critical importance for planned operation on ITER.  The work presented in this thesis (see also \cite{Walk2012}) was undertaken as a contribution to a joint research effort across several machines \cite{Groebner2013} to develop a predictive model for the ELMy H-mode pedestal.  

The empirical and computational analyses developed for ELMy H-mode are also applied in this thesis to the I-mode \cite{Whyte2010}, a novel high-confinement regime developed on Alcator C-Mod, with a number of highly desirable characteristics for reactor operation.  I-mode is notable in that it decouples energy and particle transport, forming an H-mode-like temperature pedestal with high energy confinement while maintaining an L-mode-like density profile with low particle confinement and favorably rapid transport of impurities from the plasma.  Moreover, the energy confinement in I-mode appears to degrade significantly more weakly than H-mode.  The work in this thesis (see also \cite{Walk2014}) focuses on characterizing the I-mode pedestal in terms of its structure and impact on global performance -- and therefore implications for operation on larger devices -- and its inherent avoidance of the instabilities associated with the ELM trigger.\nicesectionending

\section{Contributions to ELMy H-Mode Physics}\label{sec:conc_elmy}

ELMy H-mode experiments on C-Mod, described in \cref{ch:Elmy}, significantly expand the parameter range of investigation: C-Mod operation reaches the highest thermal pressure of any tokamak, up to within a factor of $\sim 2$ of the target pedestal pressure for ITER, as well as reaching the highest magnetic field at $\SI{8}{\tesla}$.  These experiments also entailed a significant scan in plasma current ($400-\SI{1100}{\kilo\ampere}$) and field ($3.5-\SI{8}{\tesla}$), representing a broad parameter range well outside of the explored operational space on other devices, at ITER-like conditions in several respects (\eg field, density, pressure).

The observed behavior in the pedestal are consistent with limits based in peeling-ballooning MHD instability and kinetic-ballooning turbulence (described in \cref{sec:mod_pb} and \cref{sec:mod_turbulence}, respectively).  Over the scan in plasma current (see \cref{sec:elmy_engineer}), the pressure pedestal height exhibits a trend of $p_{ped} \sim I_p$, while the pressure pedestal width scales as $\Delta_{p_e} \sim I_p^{-1}$.  Within this trend, the density and temperature pedestal widths individually exhibit no systematic trend with the plasma current, while the density and temperature pedestal heights both exhibit a weakly-positive trend with current.  This behavior is consistent with the expected $\nabla p \sim I_p^2$ constraint expected from the ballooning MHD stability limit; however, there is significant scatter, which may be corrected with a more refined analysis accounting for the pedestal width.  The attainable poloidal beta at the pedestal top is set largely by plasma shaping -- this is consistent with the ballooning MHD limit, constrained in terms of $\alpha_{MHD} \sim d\beta_{p}/d\psi$, which over the fairly restricted pedestal width sets (to zero'th order approximation) a limit on $\beta_{p,ped}$, with the attainable beta increasing with shaping due its stabilizing effect on the MHD mode.

The pedestal width is seen to scale as $\Delta \sim \beta_{p,ped}^{1/2}$, as expected from KBM turbulence (\cf \cref{subsec:elmy_eped_width}, particularly \cref{fig:elmy_betap_deltapsi_betabin}).  Alternate width models (\cref{subsec:elmy_width_oldmodels}) are largely discounted -- the density pedestal width does not exhibit the inverse dependence on the pedestal density expected from neutral-penetration models, while the temperature pedestal exhibits no dependence on the poloidal gyroradius, in contrast to predictions based on ion-orbit loss influence on the $\vec{E}\times\vec{B}$ sheared flow.  The pressure pedestal width variation with poloidal gyroradius is explained by the strong co-variance between $\rho_{i,pol}$ and $\beta_{p,ped}$.  The scatter in the pedestal height versus plasma current is corrected by accounting for this variation in width.  Taking the pedestal height to scale as $p_{ped} \sim \nabla p \times \Delta_p$, the pedestal height is predicted well by the combination of the ballooning MHD limit and the width scaling above, $p_{ped} \sim I_p^2 \beta_{p,ped}^{1/2} \sim I_p \sqrt{n_{e,ped} T_{e,ped}}$.  To lowest-order approximation, both the pedestal width and poloidal beta are robust, with both $\nabla p$ and $p_{ped}$ varying as $I_p^2$ -- the inclusion of the beta scaling accounts for much of the dependence in ELM-limited pedestals on shaping.

Strictly, the KBM model allows for secondary variation on shaping and collisionality on the mode threshold, taking the form $\Delta = G(\nu^*,\varepsilon,...) \beta_{p,ped}^{1/2}$ where $G$ is a weakly-varying function (taken to be constant in the simplest model).  The scale function is fitted to an average value $\langle G \rangle = 0.0857$, with no systematic depedence of the width on field, collisionality, shaping, or toroidal gyroradius observed (\cref{subsec:elmy_normwidth}).  

ELMy H-mode pedestals are also tested against results from the EPED model \cite{Snyder2011}, described in \cref{sec:mod_eped}, which operates based on self-consistent calculations of the peeling-ballooning MHD and KBM turbulence limits.  Notably, EPED bases its calculations only on engineering target parameters, such that it may be used predictively, rather than relying on reconstructed experimental data.  EPED predictions (see \cref{sec:elmy_eped}) for the pressure pedestal width and height match the observed results to within the $\sim 20\%$ systematic uncertainty of the model, with the best correspondence between prediction and experiment when the observed data is treated to only use time frames immediately preceding the ELM crash, when the pedestal structure most closely matches the point of instability.  Notably, the application of the EPED model to C-Mod has motivated development of the model to treat higher collisionalities and densities than typically found on other machines.\nicesectionending

\section{Contributions to I-Mode Physics}\label{sec:conc_imode}

I-mode exhibits a number of highly-desirable behaviors for reactor-scale operation -- the formation of a hot, H-mode like temperature pedestal provides good energy confinement at ITER-relevant collisionality, while the absence of a density pedestal prevents the accumulation of impurities in the plasma, a particularly important trait for machines operating with large heat fluxes onto high-$Z$ metal plasma-facing components (as is the case on C-Mod, and as expected for ITER).  Moreover, I-mode appears to lack large ELMs, avoiding the large pulsed heat loads anticipated for uncontrolled ELMy H-modes on ITER without the need for active engineering solutions for ELM mitigation or suppression.  The work presented in this thesis draws on a dedicated series of experiments across a range of plasma current, density, and heating power, prepared with high-quality pedestal profile data.  This provides a large dataset for empirical observations of pedestal structure, with the goal of improving the understanding of I-mode performance and the potential for extrapolation to higher performance.  These high-grade profiles are also useful for a detailed computational approach to the stability of the I-mode pedestal in terms of the instabilities identified with the ELM trigger, applying the stability-analysis techniques used in \cref{sec:conc_elmy,ch:Elmy} for ELMy H-mode.

\subsection{Empirical Observations}\label{subsec:conc_imode_emp}

Empirical observations of the I-mode pedestal structure and its impact on global performance, described in \cref{ch:ImodePedestal}, are illuminating in terms of the observed global behavior in I-mode.  

The temperature pedestal (\cref{subsec:imode_temp}) in I-mode exhibits some H-mode-like behavior, although the response to heating power is significantly modified.  The pedestal temperature scales roughly as $T_e \sim I_p$, albeit with significant scatter at a given current point due to variations in heating power.  This is consistent with the observed behavior in ELMy H-mode (recall that both density and temperature in ELMy H-mode are positively correlated with current, consistent with the zero'th order limit on $\beta_{p,ped}$ at consistent shaping).  Notably, the I-mode pedestal temperature meets or exceeds that found in ELMy H-mode at comparable current -- a highly beneficial trait for global performance, as high temperature pedestals support steep core temperature gradients and high core temperature and pressure (and therefore fusion power density).  At fixed current, the pedestal temperature exhibits a strong trend with heating power per particle, $T_{e,95} \sim P_{net}/\overline{n}_e$.  This is distinct from the behavior in ELMy H-mode, for which the temperature responds only weakly to heating power (rather, elevated power tends to increase ELM-driven heat transport to maintain the pedestal limit).  EDA H-modes, which are not constrained by the strong ELMing MHD limit, exhibit a positive trend of pedestal temperature with $P_{net}/\overline{n}_e$, although the sensitivity is weaker than in I-mode.

In contrast, the density profile in I-mode (\cref{subsec:imode_fueling}) exhibits markedly different behavior compared to H-mode.  The edge density in I-mode is set primarily by operator fueling (via gas puffing on C-Mod), maintaining an L-mode-like profile without a strong pedestal.  Moreover, the density profile is set largely independently of the temperature pedestal -- given sufficient power to maintain a consistent $P_{net}/\overline{n}_e$, the temperature pedestal can be matched across a range of fueling levels (see \cref{fig:imode_fuelingprofiles}).  This contrasts strongly with observed H-mode behavior -- in MHD-limited pedestals the density and temperature exhibit an inverse relationship to maintain the lowest-order $\beta_p$ limit, with the pedestal beta set by shaping, with the interplay between density and temperature instead playing a role in the ELM dynamics.  Transport-limited pedestals exhibit a similar response to heating power, but lack the ready control of the density profile, with the pedestal density instead forced by the interplay between transport and the inward particle pinch.  The L-mode-like density profile in I-mode is highly desirable for ITER scenarios, as the inward turbulent particle pinch is necessary to sufficiently fuel the core in plasmas with high edge neutral opacity (as is found on C-Mod, and as expected for ITER).

Despite the lack of a density pedestal, the I-mode pedestal (\cref{subsec:imode_pres}) reaches competitive levels of thermal pressure compared to H-mode, while maintaining the beneficial behavior in density and temperature.  Consistent with the positive trend in temperature with plasma current, the pedestal pressure scales as $p_{95} \sim I_p$, with additional scatter at each current point due to heating power and fueling levels (particularly, the pressure trends strongly with $\overline{n}_e$, visible in \cref{fig:imode_Ip_p95}).  The trend $T_{e,95} \sim P_{net}/\overline{n}_e$ implies $p_{95} \sim P_{net}$, as is observed at fixed current.  This is indicative of the weaker degradation of confinement with heating power, and is a significantly stronger response than is observed in H-mode.  It should be noted, however, that the pressure pedestal is typically relaxed compared to that in comparable H-modes, exhibiting a $\nabla p$ less than that found in H-mode, and scaling more weakly than the $\nabla p \sim I_p^2$ expected from MHD stability.

The temperature and pressure pedestal width (see \cref{sec:imode_width}) in I-mode appears to be quite robust, such that the peak temperature and pressure gradients trend linearly with their pedestal-top values.  In particular, the pressure pedestal width exhibits no systematic dependence on poloidal beta, as expected for pedestals limited by kinetic-ballooning turbulence (described in \cref{sec:conc_elmy}); The pedestal is also systematically wider than predicted by the EPED1-like width scaling.  Similarly, the width exhibits no systematic trend with poloidal gyroradius, collisionality, magnetic shear, or heat flux through the pedestal.

As the pedestal sets a strong constraint on global performance, $p_{ped} \sim W_{MHD}$ (\cref{fig:imode_p95_W}), this dataset is also suitable for explorations of the global behavior in I-mode (see \cref{sec:imode_confinement}).  Although the I-mode pressure pedestal is more relaxed than H-mode, the high pedestal temperature, combined with stiff temperature profiles in the plasma interior, supports very high core temperatures, which provides comparable core and average pressure to H-mode provided moderate density peaking, as well as comparable normalized energy confinement (see \cref{fig:imode_coreprofs,fig:imode_betan_H}).  Global stored energy reflects the strong response of the pedestal to heating power and fueling -- stored energy scales linearly with $P_{net} I_p$, consistent with weak degradation of $\tau_E$ with heating power, as well as increasing strongly with fueling, contrary to ELMy H-mode.  I-mode energy confinement is examined under a power-law fit, following the form of the ITER89 and ITER98 scalings for L- and H-mode (see \cref{subsec:imode_powerlaws}).  A simplified single-machine parameter fit reliably captures I-mode confinement over the (admittedly somewhat restricted) parameter range captures the beneficial effects of current and field on confinement, as well as the weak degradation with heating power.  As an illustrative exercise, this scaling (modified with an ITER98-like machine size dependence, which is not captured in the single-machine fit) is applied to DIII-D, ASDEX Upgrade, JET, and ITER, demonstrating the potential of I-mode operation on larger devices, with a predicted $\tau_E$ for ITER of $\sim \SI{8}{\second}$, well in excess of that expected from ITER98y2.  The degradation of $\tau_E$ with power, and plasma response to fueling, is consistent with ITER simulations supporting a method for I-mode access and approach to a $Q=10$ scenario on ITER \cite{Whyte2011}.

\subsection{Stability Modeling}\label{subsec:conc_imode_mod}

The predictive computational model utilized in \cref{ch:Elmy}, based on coupled peeling-ballooning MHD instabilities and kinetic-ballooning turbulence, are applied to the I-mode pedestal in \cref{ch:ImodeModeling} with an eye towards the observed lack of large, deleterious ELMs in the regime.  Peeling-ballooning MHD stability is calculated using the ELITE code, described in \cref{subsec:mod_elite}, while the kinetic-ballooning threshold is found using an infinite-$n$ ballooning MHD analogue calculated by BALOO (\cref{subsec:mod_balloon}).

The majority of I-mode cases are naturally free of ELMs -- these cases are modeled to be strongly stable to peeling-ballooning MHD modes.  This is intuitively consistent with the I-mode pedestal structure, in which the relaxed density profile reduces both the total pressure gradient (which was observed to be more relaxed than in ELMy H-mode) and the bootstrap current drive, stabilizing both peeling and ballooning modes.  Similarly, these cases are modeled to be below the KBM threshold (recall that the pedestal width in I-mode is observed to lack the scaling with $\beta_{p,ped}$ expected for pedestals at the KBM limit).  The pedestal parameter space in density and temperature is consistent with the calculated stability -- density and temperature (normalized to poloial field) are found to be uncorrelated, rather than exhibiting the inverse relation expected from the peeling-ballooning limit, while the pedestal poloidal beta increases linearly with normalized current, indicative of the strongly beneficial effect of fueling on pedestal and global performance.

A minority of I-mode cases, particularly at reduced current and magnetic field, have been seen to exhibit small, intermittent ELM-like events.  These events may be broken into three categories.  The majority of the events are observed to be timed with the sawtooth heat pulse reaching the pedestal, and do not appear to negatively perturb the pedestal (evidenced by the lack of a crash on the edge temperature visible on the ECE signal).  A minority of events, however, do appear to perturb the temperature pedestal; moreover, cases both triggered by the sawtooth heat pulse and independent of the sawtooth cycle are observed.  The former, sawtooth-triggered case has also been examined for peeling-ballooning MHD and KBM stability, and is found to also be stable.  The transient modification of the temperature pedestal by the sawtooth heat pulse is accounted for (see \cref{fig:imode_stbin}), and is seen to modify the pedestal in stability space, but is insufficient to reach the stability boundary.  This suggests that these cases are not edge-instability-triggered ELMs of the usual type, but rather a benign sawtooth-driven $H_\alpha$ spike, possibly driven by an ionization front in the edge neutrals from the sawtooth heat pulse without negatively perturbing the pedestal.  The latter case, despite exhibiting the expected edge behavior for the ELM, is modeled (for the stationary I-mode pedestal surrounding the ELM event) to be peeling-ballooning stable, shown in \cref{fig:imode_elite_nonstelms}, indicating transient events may be triggering the ELM instability (for which the sawtooth is an obvious candidate).  Work in quantifying the ELMs (or ELM-like events) in I-mode is ongoing, and cases with ELMs that do exhibit pedestal temperature perturbations have not been thoroughly examined for peeling-ballooning stability immediately at the ELM crash itself.\nicesectionending

\section{Future Work}\label{sec:conc_future}

The development and understanding of high-performance regimes which avoid large, deleterious ELMs -- either via physics solutions, in which the pedestal self-regulates such that the ELM boundary is not reached, or via engineering solutions, which externally apply means to suppress or mitigate ELMs -- is of crucial importance to the planning and development of operations on ITER and beyond.  Research on Alcator C-Mod has contributed greatly to this endeavor, and presents a number of potential avenues for continued research.

\subsection{ELMy H-mode}\label{subsec:conc_future_elmy}

The ELMy H-mode continues to be a valuable research path, as it is the most common shared operational regime among tokamak experiments and is considered the baseline scenario for ITER operation (although this expectation is modified by the growing necessity of control or mitigation of large type-I ELMs).  The EPED model series continues to be developed to assess this regime, as well as related regimes near the ELMing limit (\eg QH-mode, RMP-controlled H-modes).  C-Mod is an excellent test bed for this model, as it regularly operates well outside the common parameter range of other major tokamaks in terms of collisionality, field, and density, while reaching several ITER-relevant values.  In particular, the strong sensitivity of higher-collisionality pedestals on C-Mod to diamagnetic effects is necessary to test more detailed calculations of the diamagnetic stabilization threshold used in ELITE.  Proposed implementations of the EPED model operating on more generalized tokamak equilibria, rather than the up/down-symmetric Miller equilibria currently used, could also be benchmarked against previous EPED predictions on C-Mod, as the experimental shape used is quite different from the model equilibrium (indeed, the accuracy of EPED calculations is remarkable given the discrepancy between model and experimental equilibria!).

\subsection{I-mode}\label{subsec:conc_future_imode}

The desirable properties found in I-mode -- good (low) impurity confinement, high core temperature with density control, lack of large, deleterious ELMs, and lack of strong degradation in energy confinement with heating power -- have attracted significant interest in the fusion research community.  Nevertheless, there is considerable research still necessary to develop I-mode into a readily-accessible regime in the high-performance ``toolkit'' for tokamak operation.  These topics are currently high-priority research areas on C-Mod, with experiments planned on other machines as well.

While the work presented in this thesis forms a reasonable basis for the understanding of the stationary pedestal structure in I-mode, transient events and fluctuations -- that is to say, ELMs and the weakly-coherent mode (WCM) fluctuation -- may be further examined in detail.  The observed ELM behavior in I-mode are quite complex: the distinction between triggering mechanisms (\ie triggered by the sawtooth heat pulse versus spontaneous ELMs) and temperature pedestal perturbation (or lack thereof) suggests that multiple distinct phenomena may be in play, while the relative scarcity of transient ELM events in I-mode make a statistical approach difficult.  This requires thorough and carefully-diagnosed experiments in I-mode, focusing on the parameter range known to trigger these events (particularly, low toroidal field), with edge and divertor diagnostics to carefully characterize the ELM dynamics.  Further modeling efforts are also possible on this front -- while peeling-ballooning analysis of ELMing I-mode cases indicates that the stationary pedestal structure was stable (see \cref{fig:imode_elite_nonstelms}), this analysis included substantial non-ELMing periods, leaving transient pedestal instabilities as a possibility.  A more careful treatment of the KBM is also possible, accounting for the reduction of the critical gradient from the ideal-MHD result by ion-drift-resonance effects in steep temperature profiles; significant modification to the KBM threshold is seen even at $\eta_i = L_{T_i}/L_{n_i} = 2$ (a reasonable value in H-mode) \cite{Snyder2001}, such that I-mode pedestals with $\eta_i \sim 10$ \cite{Whyte2010} may exhibit a KBM threshold significantly different from the ideal-ballooning surrogate.

Given its apparent role in regulating the density profile in I-mode, a clear understanding of the WCM is desirable.  Planned experiments in the upcoming run campaign on C-Mod will explore the WCM fluctuation amplitude, radial extent, and localization in the pedestal.  This data can be integrated into the existing ``pedestal database'' structure used in this thesis for ready analysis in terms of pedestal profiles and engineering parameters.  Modeling of the WCM in gyrokinetic and fluid codes (\eg GYRO, BOUT++) is also planned.

Although I-mode operation has demonstrated good performance on C-Mod, its density is still restricted below that found in H-modes.  In a burning plasma, heating power is set by the fusion reaction rate, proportional to $n_e^2$ -- as such, I-mode operation with increased operational window in density is an ongoing effort on C-Mod.  The results presented in this thesis are promising in this regard, as the application of matched increases in fueling in heating power (such that the temperature pedestal is maintained with consistent $P_{net}/\overline{n}_e$) strongly increase the pedestal pressure and global beta/stored energy, while maintaining the beneficial relaxed edge density gradient.  The question remains whether large-scale (\eg high-performance targets on ITER) can reach the necessary $P_{net}/\overline{n}_e$ to sustain the target temperature pedestal, necessitating further study for pedestal optimization.  The impact of increased fueling on pedestal stability will also be examined -- while initial observations indicate that the density gradient remains relaxed at higher fueling levels (raising the SOL density apace with the pedestal density, rather than developing a density pedestal), the impact of increased fueling on the pressure gradient and bootstrap current drives may ultimately pose a peeling-ballooning MHD limit on the I-mode pedestal.  On the other hand, the elevated SOL density is beneficial for power handling and divertor detachment; this, coupled with the relatively broad SOL heat-flux channel observed in I-mode, may well add divertor power handling to the list of benefits to I-mode operation.

Finally, research into I-mode access on existing larger devices is still in the initial stages -- while I-mode experiments have been carried out on DIII-D and ASDEX Upgrade (with some I-mode-like features observed on JET as well) the regime has not been firmly established outside of C-Mod.  Due to the significant differences in parameter space (magnetic field, size, and density, as well as edge neutral behavior on carbon-walled devices) between C-Mod and other major tokamak experiments, the window for I-mode access on other devices is not well understood despite the relative ease of access on C-Mod.  Further experiments better characterizing the L-I and I-H thresholds on C-Mod, as well as a firm understanding of the edge fluctuation behavior (described above), are necessary to solidify I-mode access on existing tokamak experiments and projection to ITER operation.\nicechapterending

\bibliographystyle{../plainurl}
\bibliography{../references}