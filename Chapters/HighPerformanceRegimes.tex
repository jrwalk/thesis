\chapter{High-Performance Regimes}\label{ch:HighPerformance}

The development of magnetic-confinement fusion into an economical form of power generation is characterized by two seemingly contrary requirements: first, a high level of energy confinement is necessary to reach the desired level of self-heating of the plasma by fusion products, satisfying triple-product requirements (\cref{eq:tripleproduct}).  At the same time, particle transport must be sufficient to avoid the deleterious effects of accumulated helium ``fusion ash'' and other impurities on fusion performance -- particularly important in the case of the high-$Z$ impurities from the metal plasma-facing walls necessary for reactor-scale devices \cite{Loarte2007}\gnote{more cites?}.

A number of operating scenarios, collectively termed ``high confinement'' or H-modes\cite{Wagner1982}, satisfying these requirements have been developed.  The ``low confinement'' or L-mode operating baseline of energy confinement is characterized through an extensive multi-machine scaling study \cite{Yushmanov1990} by the ITER-89 scaling,

\begin{equation}\label{eq:tau89}
 \tau_{E,ITER89} = 0.048 \times \overline{n}_e^{0.1} M^{0.5} I_p^{0.85} R^{1.2} a^{0.3} \kappa^{0.5} B_T^{0.2} P_{aux}^{-0.5}
\end{equation}

\noindent in which $\overline{n}_e$ is the line-averaged density ($\SI{e20}{\per\meter\cubed}$), $M$ is the atomic mass ($\si{amu}$), $I_p$ is the plasma current ($\si{\mega\ampere}$), $B_T$ is the toroidal field ($\si{T}$), $R$ and $a$ are the major and minor radii in $\si{\meter}$ (see \cref{fig:intro_geometry}), $\kappa$ is the elongation (see \cref{fig:intro_shaping}), and $P_{aux}$ is the externally-applied heating power ($\si{\mega\watt}$).  Compared to this baseline, H-modes represent a significant improvement in performance, with confinement -- here represented in a normalized sense by the $H$-factor, \ie

\begin{equation}\label{eq:H89}
 H_{89} = \frac{\tau_E}{\tau_{E,ITER89}}
\end{equation}

\noindent improved by roughly a factor of two compared to L-mode.\gnote{more specific, cites?}

This improvement in confinement is due to the formation of a \emph{pedestal}, a transport barrier (see \cref{subsec:intro_barriers}) at the edge that greatly slows the transport of particles and/or energy out of the plasma, and accordingly forms a steep-gradient region in density and/or temperature at the edge.  Pedestal formation is achieved through strongly sheared flows in the plasma edge, driven in part by a radial electric field (the ``$E_r$ well'') and the resulting $\vec{E} \times \vec{B}$ flows in the pedestal.  While this flow is difficult to model due to the short scale lengths inherent to the pedestal \cite{Kagan2010,Landreman2012}, the role of edge $E_r$ and flows has been extensively studied both from an experimental \cite{Groebner1990,Burrell1999,Terry2000,McDermott2009a} and a theoretical \cite{Shaing1989,Biglari1990,Kim1991,Ware1996} standpoint, as has the role of other edge fluctuations coupling into these flows in driving the transition into H-mode \cite{Schmitz2012}.  As the pedestal structure is known to set a strong constraint on the overall performance in high-confinement regimes \cite{Kinsey2011}, as well as determining the edge stability and heat exhaust properties of the regime, a firm understanding of the pedestal is essential for extrapolation of a high-performance regime to ITER and beyond.

This chapter provides an overview and comparison of different classes of established H-mode operation, particularly regarding their behaviors in the high energy confinement and low particle confinement required for a reactor.  Additionally, observations of Edge-Localized Modes (ELMs) \cite{Zohm1996} are addressed.  We then introduce the access conditions, operation, and global characteristics of \emph{I-mode} -- an alternate high-performance regime with a number of favorable characteristics for reactor operation, and the subject of the balance of this thesis.

\begin{table*}[h]
 \pushtooutside
 \ttabbox{\caption{Typical operating parameters of tokamaks noted in this thesis, along with references to overviews of each machine.  \emph{Note:} all ITER values are projected.\note{check all values here}}\label{tab:tokamaks}}
 {\begin{tabular}{lcccccc}
  \toprule
  \emph{Device} &
  $R/a \;[\si{\meter}]$ &
  %$a \;[\si{\meter}]$ &
  $I_p\;[\si{\mega\ampere}]$ &
  $B_T \;[\si{\tesla}]$ &
  $\overline{n}_e \;[\si{\per\meter\cubed}]$ &
  $T_{e0} \;[\si{\kilo\electronvolt}]$ &
%   heating &
  \emph{refs}
  \\
  \midrule
  C-Mod (USA) &
  $\num{0.67/0.22}$ &
  %$\num{0.22}$ &
  $\le \num{2}$ &
  $3-8.1$ &
  $\le \num{5e20}$ &
  $\le \num{8}$ &
%   ICRF, LHRF &
  \cite{Hutchinson1994,Greenwald2007,Greenwald2013}
  \\
  DIII-D (USA) &
  $\num{1.67/0.67}$ &
  %$\num{0.67}$ &
  $1-3$ &
  $2.2$ &
  $\num{6e19}$ &
  $5-10$ &
%   ECRF, NBI &
  \cite{Luxon2002,Luxon2005a,Luxon2005}
  \\
  ASDEX-U (GER) &
  $\num{1.65/0.5}$ &
  %$\num{0.5}$ &
  $\sim 1$ &
  $3.9$ &
  $\num{7.5e19}$ &
  $2-3$ &
%   &
  \cite{Herrmann2003,Ryter2003,Stroth2013}
  \\
  JET (UK) &
  $\num{3.4/0.9}$ &
  $3-4$ &
  $3.8$ &
  $\num{5e19}$ &
  $10-20$ &
%   &
  \cite{McDonald2008,Romanelli2013}
  \\
  JT-60U (JAP) &
  $\num{3.4/0.9}$ &
  $3-4$ &
  $4.8$ &
  $\num{5e19}$ &
  $10-20$ &
%   &
  \cite{Kamada2002,Kitsunezaki2002}
  \\
  JFT-2M (JAP) &
  $\num{1.3/0.35}$ &
  $0.5$ &
  $2.2$ &
  $\num{5e19}$ &
  $1-2$ &
%   &
  \cite{Kusama2006,Miura2006}
  \\
  ITER* &
  $\num{6.2/2.0}$ &
  $15$ &
  $5.3$ &
  $\num{1e20}$ &
  $10$ &
%   &
  \cite{Shimada2007,ITER1999}
  \\
  \bottomrule
 \end{tabular}}
\end{table*}

\section{ELM-Free H-Mode}\label{sec:hcr_elmfree}

\nicesectionending

\section{ELMy H-Mode}\label{sec:hcr_elmy}

\subsection{Global Parameters}\label{subsec:hcr_elmy_ped}

\subsection{Fluctuations and ELMs}\label{subsec:hcr_elmy_fluct}

\subsection{Active ELM Control}\label{subsec:hcr_elmy_control}

\nicesectionending

\section{ELM-Suppressed H-Modes}\label{sec:hcr_elmsuppressed}

In addition to H-modes exhibiting ELMs, classes of H-mode have been established capable of stationary operation with acceptable levels of particle transport (avoiding the radiative collapse and subsequent transient nature found in classical ELM-free H-modes) without exhibiting the bursty heat and particle transport driven by ELMs.  Rather, the pedestal is regulated by a continuous fluctuation localized in the pedestal.  Due to this attractive property, these regimes have been extensively researched\gnote{reword}.  The characteristics of two major types, the Quiescent H-mode (QH-mode) and Enhanced $D_\alpha$ (EDA) H-mode are presented here.

\subsection{QH-Mode}\label{subsec:hcr_qh}

The \emph{Quiescent H-mode} (QH-mode) was first observed on DIII-D \cite{Burrell2002,Groebner2001}, and subsequently achieved on ASDEX Upgrade \cite{Suttrop2003a}, JT-60U \cite{Sakamoto2004}, and JET \cite{Suttrop2005}.  In QH-mode operation, following a brief ELM-free or ELMing phase after the L-H transition, the plasma enters a state with steady averaged density and radiated power, indicating a lack of serious impurity accumulation, despite lacking ELM transport (evident from divertor $D_\alpha$ light, which is ``quiescent'' compared to the characteristic spikes driven by ELMs).  Although QH-mode requires lower densities (average density reduced by roughly a factor of two from comparable ELMy H-modes) with cryopumping for density control, access is otherwise robust, with successful operation across a broad range of shaping, safety factor, current and field \cite{Burrell2002}.  The regime is capable of stationary operation, with the mode sustained for most of the current flat-top on DIII-D ($\sim 25 \tau_E$) with very good confinement -- in cases with an internal transport barrier in addition to the pedestal (termed the ``Quiescent Double Barrier'' or QDB regime \cite{Burrell2001,Doyle2001,Greenfield2002}) a confinement metric of $\beta_N H_{89} \sim 7$ was reached (albeit for a briefer period, $\sim 5 \tau_E$), compared to $\beta_N H_{89} \sim 4$ found in ELMy H-modes on DIII-D \cite{Doyle2001}.  Here we use for a normalized pressure metric.\gnote{cite for this?}

\begin{equation}\label{eq:betan}
 \beta_N = \beta \frac{aB_T}{I_p}
\end{equation}

\noindent in $\si{\meter.\tesla\per\mega\ampere}$.  Similarly competitive confinement between QH-mode and ELMy H-mode is seen on ASDEX Upgrade and JET, although the mode on JT-60U is out-performed by ELMy H-mode \cite{Oyama2006}.  The pedestal density is reduced (comparable to the reduction in globally-averaged density) in QH-mode compared to ELMy H-mode, and excess fueling to the edge by gas puffing, pellet fueling, or wall outgassing destroys the QH-mode.  However, pedestal temperatures are typically somewhat higher \cite{Doyle2001}, thus the mode is found at ITER-relevant low collisionalities.  Pedestal pressure gradients are comparable to those found in ELMy H-mode, implying stabilization of the peeling-ballooning MHD modes typically associated with the ELM trigger \cite{Burrell2002}.\gnote{elaborate?}  A particularly strong $E_r$ well ($2-3$ times deeper than in comparable ELMy H-modes) is also observed in the QH-mode pedestal \cite{Greenfield2002}.

In place of bursty ELM transport, the pedestal in QH-mode is continuously regulated by the Edge Harmonic Oscillation (EHO) an MHD mode observed in density, temperature, and magnetic fluctuations \cite{Burrell2002}.  The EHO is made up of distinct harmonics with toroidal mode numbers $n \sim 1-10$; these harmonics are directly observed in the particle flux at the divertor, indicating that the EHO is responsible for density regulation in QH-mode \cite{Doyle2001}.  MHD modeling approaches similar to that described in \cref{ch:Modeling} indicate that the EHO is a saturated peeling mode \cite{Snyder2007,Osborne2008}.  This is consistent with the low pedestal collisionality in the QH-mode pedestal (lower collisionalities and higher bootstrap currents tends to drive the pedestal towards the peeling side of the peeling-ballooning MHD boundary, as described in \cref{ch:Modeling}), and with the observed localization of the EHO in the region of strongest $E_r$ and rotation shear \cite{Burrell2001}.  the saturated mode is driven by the strong rotation shear in the edge -- while this typically destabilizes low-$n$ MHD modes\gnote{pull ref 9 from Burrell2009?}, in the case of the EHO the magnetic component of the mode couples to the vacuum-vessel wall as the rotation spins up, providing the drag force necessary to saturate the mode at finite amplitude \cite{Burrell2009}.  This maintains the pedestal below the current-driven peeling boundary associated with the ELM trigger, providing the ELM suppression in QH-mode\gnote{recent ref from Phil for this?}.

Historically, QH-mode operation has required significant neutral-beam inputs directed counter the plasma current direction, providing the necessary rotation \cite{Burrell2002}.  However, counter-current beam operation drives significant fast-ion losses into the outer wall, necessitating operation with a large outer gap to avoid wall outgassing.  More recent experiments have successfully generated QH-modes with co-current beam injection \cite{Burrell2009} and with torque from non-axisymmetric magnetic fields \cite{Garofalo2011,Burrell2013}.  The latter is of particular importance, as it is not expected that the NBI systems on ITER will drive sufficient torque to produce QH-mode \cite{Garofalo2011}.  In addition to the requirement for externally-supplied torque to maintain the mode, QH-mode suffers from accumulation of high-$Z$ impurities -- while lower-$Z$ ions are flushed from the plasma by the EHO, high-$Z$ impurities tend to accumulate in the core \cite{Doyle2001,Suttrop2005}, which may present difficulties attaining QH-mode on metal-walled machines where high-$Z$ impurities dominate.  Nevertheless QH-mode is an attractive option for a reactor regime.

\subsection{EDA H-Mode}\label{subsec:hcr_eda}

\nicesectionending

\section{I-Mode}\label{sec:hcr_imode}

\subsection{Access and Operation}\label{subsec:hcr_imode_access}

\subsection{Global Performance}\label{subsec:hcr_imode_performance}

\subsection{Edge Fluctuations -- the Weakly-Coherent Mode}\label{subsec:hcr_imode_wcm}

\nicechapterending

\bibliographystyle{../plainurl}
\bibliography{../references}