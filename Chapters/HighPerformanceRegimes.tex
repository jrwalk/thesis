\chapter{High-Performance Regimes}\label{ch:HighPerformance}

The development of magnetic-confinement fusion into an economical form of power generation is characterized by two seemingly contrary requirements: first, a high level of energy confinement is necessary to reach the desired level of self-heating of the plasma by fusion products, satisfying triple-product requirements (\cref{eq:tripleproduct}).  At the same time, particle transport must be sufficient to avoid the deleterious effects of accumulated helium ``fusion ash'' and other impurities on fusion performance -- particularly important in the case of the high-$Z$ impurities from the metal plasma-facing walls necessary for reactor-scale devices \cite{Loarte2007}\gnote{more cites?}.

A number of operating scenarios, collectively termed ``high confinement'' or H-modes\cite{Wagner1982}, satisfying these requirements have been developed.  The ``low confinement'' or L-mode operating baseline of energy confinement, characterized through an extensive multi-machine scaling study \cite{Yushmanov1990} by the ITER-89 scaling,

\begin{equation}\label{eq:tau89}
 \tau_{E,ITER89} = 0.048 \times \overline{n}_e^{0.1} M^{0.5} I_p^{0.85} R^{1.2} a^{0.3} \kappa^{0.5} B_T^{0.2} P_{aux}^{-0.5}
\end{equation}

\noindent in which $\overline{n}_e$ is the line-averaged density ($\SI{e20}{\per\meter\cubed}$), $M$ is the atomic mass ($\si{amu}$), $I_p$ is the plasma current ($\si{\mega\ampere}$), $B_T$ is the toroidal field ($\si{T}$), $R$ and $a$ are the major and minor radii in $\si{\meter}$ (see \cref{fig:intro_geometry}), $\kappa$ is the elongation (see \cref{fig:intro_shaping}), and $P_{aux}$ is the externally-applied heating power ($\si{\mega\watt}$).  Compared to this baseline, H-modes represent a significant improvement in performance, with confinement -- here represented in a normalized sense by the $H$-factor, \ie

\begin{equation}\label{eq:H89}
 H_{89} = \frac{\tau_E}{\tau_{E,ITER89}}
\end{equation}

\noindent by at least a factor of two.\gnote{more specific, cites?}

\begin{table*}[h]
 \pushtooutside
 \ttabbox{\caption{Typical operating parameters of tokamaks noted in this thesis, with reference to overviews.  \emph{Note:} all ITER values are projected.\note{check all values here}}\label{tab:tokamaks}}
 {\begin{tabular}{lcccccc}
  \toprule
  \emph{Device} &
  $R/a \;[\si{\meter}]$ &
  %$a \;[\si{\meter}]$ &
  $I_p\;[\si{\mega\ampere}]$ &
  $B_T \;[\si{\tesla}]$ &
  $\overline{n}_e \;[\si{\per\meter\cubed}]$ &
  $T_{e0} \;[\si{\kilo\electronvolt}]$ &
%   heating &
  \emph{refs}
  \\
  \midrule
  C-Mod (USA) &
  $\num{0.67/0.22}$ &
  %$\num{0.22}$ &
  $\le \num{2}$ &
  $3-8.1$ &
  $\le \num{5e20}$ &
  $\le \num{8}$ &
%   ICRF, LHRF &
  \cite{Hutchinson1994,Greenwald2007,Greenwald2013}
  \\
  DIII-D (USA) &
  $\num{1.67/0.67}$ &
  %$\num{0.67}$ &
  $1-3$ &
  $2.2$ &
  $\num{6e19}$ &
  $5-10$ &
%   ECRF, NBI &
  \cite{Luxon2002,Luxon2005a,Luxon2005}
  \\
  ASDEX-U (GER) &
  $\num{1.65/0.5}$ &
  %$\num{0.5}$ &
  $\sim 1$ &
  $3.9$ &
  $\num{7.5e19}$ &
  $2-3$ &
%   &
  \cite{Herrmann2003,Ryter2003,Stroth2013}
  \\
  JET (UK) &
  $\num{3.4/0.9}$ &
  $3-4$ &
  $3.8$ &
  $\num{5e19}$ &
  $10-20$ &
%   &
  \cite{McDonald2008,Romanelli2013}
  \\
  JT-60U (JAP) &
  $\num{3.4/0.9}$ &
  $3-4$ &
  $4.8$ &
  $\num{5e19}$ &
  $10-20$ &
%   &
  \cite{Kamada2002,Kitsunezaki2002}
  \\
  JFT-2M (JAP) &
  $\num{1.3/0.35}$ &
  $0.5$ &
  $2.2$ &
  $\num{5e19}$ &
  $1-2$ &
%   &
  \cite{Kusama2006,Miura2006}
  \\
  ITER* &
  $\num{6.2/2.0}$ &
  $15$ &
  $5.3$ &
  $\num{1e20}$ &
  $10$ &
%   &
  \cite{Shimada2007,ITER1999}
  \\
  \bottomrule
 \end{tabular}}
\end{table*}

\section{ELM-Free H-Mode}\label{sec:hcr_elmfree}

\nicesectionending

\section{ELMy H-Mode}\label{sec:hcr_elmy}

\subsection{Global Parameters}\label{subsec:hcr_elmy_ped}

\subsection{Fluctuations and ELMs}\label{subsec:hcr_elmy_fluct}

\subsection{Active ELM Control}\label{subsec:hcr_elmy_control}

\nicesectionending

\section{ELM-Suppressed H-Modes}\label{sec:hcr_elmsuppressed}

\subsection{QH-Mode}\label{subsec:hcr_qh}

\subsection{EDA H-Mode}\label{subsec:hcr_eda}

\nicesectionending

\section{I-Mode}\label{sec:hcr_imode}

\subsection{Access and Operation}\label{subsec:hcr_imode_access}

\subsection{Global Performance}\label{subsec:hcr_imode_performance}

\subsection{Edge Fluctuations -- the Weakly-Coherent Mode}\label{subsec:hcr_imode_wcm}

\nicechapterending

\bibliographystyle{../plainurl}
\bibliography{../references}