%*******************************************************
% Acknowledgments
%*******************************************************

\chapter*{Acknowledgments}

The work presented here is the culmination of years of work in graduate school, as well as in undergrad -- along the way, a great many people have helped to get me to where I stand now.  There have certainly been too many people to name, but that won't stop me from trying anyway.

This thesis would have been utterly impossible without the support of my committee.  First, I wish to thank my advisor, Dr. Jerry Hughes -- a man who combines experimental acumen and good humor, and who has proven an excellent mentor even in the face of my incessant questions, usually right before (or right after) a deadline.  He has been a pleasure to work with, and I look forward to continuing to do so in the future.  I am no less grateful to Prof. Dennis Whyte, the reader of this thesis.  He has been a constant source of insight, particularly in the ``big picture'' for fusion research.  The committee was rounded out by Prof. Anne White, who (among her many contributions) deserves recognition for her constant support of student efforts on C-Mod and in the NSE department here at MIT.

A number of other scientists merit special mention as well.  First, Phil Snyder of General Atomics, whose theoretical work and development of the ELITE and EPED codes contributed much to the analysis presented in this thesis.  He possesses a rare gift for connecting the esoterics of theory to experiment.  Amanda Hubbard is due thanks for her tireless work in I-mode experiments, and for her perpetually open door for the many questions I had during this thesis.  None of the work here would have happened without the help of John Rice, who took an MIT sophomore and set him on the path to this thesis.

Odds are, if you've worked on Alcator C-Mod in the last four years, I've come to you at some point for help.  None of the experiments in this thesis would have been possible without the army of research scientists, engineers, and technicians who keep C-Mod running.  Among others: Jim Terry, Brian LaBombard, Luis Delgado-Aparicio, Steve Wolfe, Steve Wukitch, Bob Granetz, Yijun Lin, Atma Kanojia, Bruce Lipschultz, Paul Bonoli, Ron Parker, Ian Hutchinson, Greg Wallace... all have contributed data, PhysOp time, and discussion in the control room, and have helped make the PSFC an incredible place to work.  The entire technical team on C-Mod is due thanks for tireless work running the machine -- in particular, I want to thank Tom Toland and Ron Rosati for many long hours spent on calibrations for the Thomson Scattering system.  On the administrative side, thanks to Jessica Coco and Valerie Censabella for undertaking the impossible task of keeping a building full of physicists in some semblance of order -- without you, I fully expect the PSFC would devolve into some sort of \emph{Lord of the Flies} scenario, pig's head on a stick, fat kid crushed under a rock, the works.  Likewise, our senior leadership -- Earl Marmar and Martin Greenwald for C-Mod, and Miklos Porkolab for the PSFC -- deserve credit for keeping this lab up and running and converting caffeine to science.

C-Mod wouldn't be C-Mod without the host of grad students (past and present) and postdocs who have made my time here memorable: Seung Gyou Baek, Harold Barnard, Dan Brunner, Mark Chilenski, Mike Churchill, Istvan Cziegler, Evan Davis, Paul Ennever, Ian Faust, Chi Gao, Mike Garrett, Ted Golfinopoulos, Christian Haakonsen, Zach Hartwig, Nathan Howard, Alex Ince-Cushman, Leigh Ann Kesler, Cornwall Lao, Jungpyo Lee, Yunxing Ma, Orso Meneghini, Bob Mumgaard, Roman Ochoukov, Geoff Olynyk, Yuri Podpaly, Matt Reinke, Jude Safo, Jen Sierchio, Brandon Sorbom, Choongki Sung, and Christian Theiler.  In particular, I want to thank Ian Faust, with whom I've shared an office (including the infamous Winchester, bead door and all) -- you've racked up some serious karma for putting up with my various oddities.  I also want to add a special mention for Mark Chilenski's work on Python tools for C-Mod, both the \emph{eqtools} set developed by Mark, Ian, and myself, and for his solo work on the \emph{gptools} and \emph{profiletools} packages (which, among other things, contributed density-peaking analysis to the I-mode data in this thesis).

The experimental work in this thesis was completed on the Alcator C-Mod tokamak, a DOE Office of Science user facility.  Work at Alcator C-Mod is supported by US DOE Agreement No. DE-FC02-99ER54512.  Theory work at General Atomics is supported by US DOE Agreement No. DE-FG02-99ER54309.