%\twocolumn[
%  \begin{@twocolumnfalse}

High-performance operation in tokamaks is characterized by the formation of a \emph{pedestal}, a region of suppressed transport and steep gradients in density, temperature, and pressure near the plasma edge.  The pedestal height is strongly correlated with overall fusion performance, as a substantial pedestal supports the elevated core pressure necessary for the desired fusion reaction rate and power density.  However, stationary operation requires some relaxation of the particle transport barrier, to avoid the accumulation of impurities (\eg helium ``fusion ash,'' plasma-facing surface materials) in the plasma.  Moreover, the formation of the pedestal introduces an additional constraint: the steep gradients act as a source of free energy for Edge-Localized Mode (ELM) instabilities, which in on ITER- or reactor-scale devices can drive large, explosive bursts of particle and energy transport leading to unacceptable levels of heat loading and erosion damage to plasma-facing materials.  As such, the suppression, mitigation, or avoidance of large ELMs is the subject of much current research.
   
In light of this, a firm physical understanding of the pedestal structure and stability against the ELM trigger is essential for the extrapolation of high-performance regimes to large-scale operation, particularly in operating scenarios lacking large, deleterious ELMs.  This thesis focuses on the \emph{I-mode}, a novel high-performance regime pioneered on the Alcator C-Mod tokamak.  I-mode is unique among high-performance regimes in that it appears to decouple energy and particle transport, reaching H-mode levels of energy confinement with the accompanying temperature pedestal while maintaining a L-mode-like density profile and particle transport.  I-mode exhibits three attractive properties for a reactor regime: (1) I-mode appears to be inherently free of large ELMs, avoiding the need for externally-applied ELM control.  (2) The lack of a particle transport barrier maintains the desired level of impurity flushing from the plasma, avoiding excessive radiative losses.  (3) Energy confinement in I-mode presents minimal degradation with input heating power, contrary to that found in H-mode.
   
This thesis presents the results from a combined empirical and computational study of the pedestal on C-Mod.  Analysis methods are first implemented in ELMy H-mode base cases on C-Mod -- in particular, the EPED model based on the combined constraints from peeling-ballooning MHD instability and kinetic-ballooning turbulence is tested on C-Mod.  Empirical results in ELMy H-mode are consistent with the physics assumptions used in EPED, with the pedestal pressure gradient constrained by $\nabla p \sim I_p^2$ expected from the ballooning stability limit.  To lowest-order approximation, ELMy H-mode pedestals are limited in $\beta_{p,ped}$, with the attainable beta set by shaping -- within this limit, an inverse relationship between pedestal density and temperature is seen.  The pedestal width is found to be described by the scaling  $\Delta_\psi = G \beta_{p,ped}^{1/2}$ expected from the KBM limit, where $G(\nu,\varepsilon,...)$ is a weakly varying function with $\langle G \rangle = 0.0857$.  No systematic secondary scalings with field, gyroradius, shaping, or collisionality are observed.  The EPED model, based on these assumptions, correctly predicts the pressure pedestal width and height to within a systematic $\sim 20\%$ uncertainty.

Empirical scalings in I-mode highlight the operational differences from conventional H-modes.  The temperature and pressure pedestal exhibit a positive trend with current, similar to H-mode (although I-mode pedestal temperature typically exceeds that found in comparable H-modes) -- however, the temperature and pressure respond significantly more strongly to heating power, with $T_{e,95} \sim P_{net}/\overline{n}_e$ and $p_{95} \sim P_{net}$.  The I-mode density profile is set largely independently of the temperature pedestal (unlike ELMy H-mode), controlled by operator fueling.  Given sufficient heating power to maintain a consistent $P_{net}/\overline{n}_e$, temperature pedestals are matched across a range of fueling levels.  This indicates a path to readier access and increased performance in I-mode, with the mode accessed at moderate density and power, after which the pedestal pressure is elevated with matched increases in fueling and heating power.  Global performance metrics in I-mode are competitive with H-mode results on C-Mod, and are consistent with the weak degradation of energy confinement with heating power.

I-mode pedestals are also examined against the physics basis for the EPED model.  Peeling-ballooning MHD stability is calculated using the ELITE code, finding the I-mode pedestal to be strongly stable to the MHD modes associated with the ELM trigger.  Similarly, 

%  \end{@twocolumnfalse}
%]
