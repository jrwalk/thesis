% ****************************************************************************************************
% classicthesis-config.tex
% formerly known as loadpackages.sty, classicthesis-ldpkg.sty, and classicthesis-preamble.sty
% Use it at the beginning of your ClassicThesis.tex, or as a LaTeX Preamble
% in your ClassicThesis.{tex,lyx} with % ****************************************************************************************************
% classicthesis-config.tex
% formerly known as loadpackages.sty, classicthesis-ldpkg.sty, and classicthesis-preamble.sty
% Use it at the beginning of your ClassicThesis.tex, or as a LaTeX Preamble
% in your ClassicThesis.{tex,lyx} with % ****************************************************************************************************
% classicthesis-config.tex
% formerly known as loadpackages.sty, classicthesis-ldpkg.sty, and classicthesis-preamble.sty
% Use it at the beginning of your ClassicThesis.tex, or as a LaTeX Preamble
% in your ClassicThesis.{tex,lyx} with % ****************************************************************************************************
% classicthesis-config.tex
% formerly known as loadpackages.sty, classicthesis-ldpkg.sty, and classicthesis-preamble.sty
% Use it at the beginning of your ClassicThesis.tex, or as a LaTeX Preamble
% in your ClassicThesis.{tex,lyx} with \input{classicthesis-config}
% ****************************************************************************************************
% If you like the classicthesis, then I would appreciate a postcard.
% My address can be found in the file ClassicThesis.pdf. A collection
% of the postcards I received so far is available online at
% http://postcards.miede.de
% ****************************************************************************************************

% ****************************************************************************************************
% 1. Configure classicthesis for your needs here, e.g., remove "drafting" below
% in order to deactivate the time-stamp on the pages
% ****************************************************************************************************
%\PassOptionsToPackage{eulerchapternumbers,%drafting,%
%				 pdfspacing,%floatperchapter,%linedheaders,%
%				 subfig,beramono,eulermath}{classicthesis}		

\PassOptionsToPackage{eulerchapternumbers,drafting,%
				 pdfspacing,%floatperchapter,%linedheaders,%
				 subfig,eulermath}{classicthesis}									
% ********************************************************************
% Available options for classicthesis.sty
% (see ClassicThesis.pdf for more information):
% drafting
% parts nochapters linedheaders
% eulerchapternumbers beramono eulermath pdfspacing minionprospacing
% tocaligned dottedtoc manychapters
% listings floatperchapter subfig
% ********************************************************************

% ********************************************************************
% Triggers for this config
% ********************************************************************
\usepackage{ifthen}
\newboolean{enable-backrefs} % enable backrefs in the bibliography
\setboolean{enable-backrefs}{false} % true false
% ****************************************************************************************************


\usepackage{geometry}%[showframe] allows differnt margins on title page, DB

% ****************************************************************************************************
% 2. Personal data and user ad-hoc commands
% ****************************************************************************************************
\newcommand{\myTitle}{A Classic Thesis Style\xspace}
\newcommand{\mySubtitle}{An Homage to The Elements of Typographic Style\xspace}
\newcommand{\myDegree}{Doktor-Ingenieur (Dr.-Ing.)\xspace}
\newcommand{\myName}{Andr\'e Miede\xspace}
\newcommand{\myProf}{Put name here\xspace}
\newcommand{\myOtherProf}{Put name here\xspace}
\newcommand{\mySupervisor}{Put name here\xspace}
\newcommand{\myFaculty}{Put data here\xspace}
\newcommand{\myDepartment}{Put data here\xspace}
\newcommand{\myUni}{Put data here\xspace}
\newcommand{\myLocation}{Darmstadt\xspace}
\newcommand{\myTime}{August 2012\xspace}
\newcommand{\myVersion}{version 4.1\xspace}

% ********************************************************************
% Setup, finetuning, and useful commands
% ********************************************************************
\newcounter{dummy} % necessary for correct hyperlinks (to index, bib, etc.)
\newlength{\abcd} % for ab..z string length calculation
\providecommand{\mLyX}{L\kern-.1667em\lower.25em\hbox{Y}\kern-.125emX\@}
\newcommand{\ie}{i.\,e.,\;}
\newcommand{\Ie}{I.\,e.,\;}
\newcommand{\eg}{e.\,g.,\;}
\newcommand{\Eg}{E.\,g.,\;}
% ****************************************************************************************************


% ****************************************************************************************************
% 3. Loading some handy packages
% ****************************************************************************************************
% ********************************************************************
% Packages with options that might require adjustments
% ********************************************************************
% \PassOptionsToPackage{utf8}{inputenc}	% latin9 (ISO-8859-9) = latin1+"Euro sign"
%  \usepackage{inputenc}				

% \PassOptionsToPackage{russian,american}{babel}   % change this to your language(s)
% % Spanish languages need extra options in order to work with this template
% %\PassOptionsToPackage{spanish,es-lcroman}{babel}
%  \usepackage{babel}					

\PassOptionsToPackage{square,numbers,sort}{natbib}
 \usepackage{natbib}				

\PassOptionsToPackage{fleqn}{amsmath}		% math environments and more by the AMS
 \usepackage{amsmath}

% ********************************************************************
% General useful packages
% ********************************************************************
\PassOptionsToPackage{T1}{fontenc} % T2A for cyrillics
	\usepackage{fontenc}
\PassOptionsToPackage{utf8}{inputenc}	% latin9 (ISO-8859-9) = latin1+"Euro sign"
	\usepackage{inputenc}	
\usepackage{textcomp} % fix warning with missing font shapes
\usepackage{scrhack} % fix warnings when using KOMA with listings package
\usepackage{xspace} % to get the spacing after macros right
\usepackage{mparhack} % get marginpar right
\usepackage{fixltx2e} % fixes some LaTeX stuff
\usepackage[detect-all]{siunitx}%convenient units, DB
\sisetup{mode=text,range-phrase = {\text{~to~}}}%allows ranges to be used in math%DB
\usepackage{setspace}%allows double spacing for draft, DB
\usepackage{bm}%allows bold math, DB
%\usepackage{mcaption}%allows captions in margins, DB
\PassOptionsToPackage{printonlyused,smaller}{acronym}
	\usepackage{acronym} % nice macros for handling all acronyms in the thesis
%\renewcommand*{\acsfont}[1]{\textssc{#1}} % for MinionPro
\renewcommand{\bflabel}[1]{{#1}\hfill} % fix the list of acronyms
\usepackage{xcolor}

\PassOptionsToPackage{russian,american}{babel}   % change this to your language(s)
% Spanish languages need extra options in order to work with this template
%\PassOptionsToPackage{spanish,es-lcroman}{babel}
 \usepackage{babel}
% ****************************************************************************************************
%
% Some new commands that I like, DB
%
\renewcommand{\sup}[1]{\ensuremath{^{\mathrm{#1}}}}%DB
\newcommand  {\sub}[1]{\ensuremath{_{\mathrm{#1}}}}%DB
\newcommand  {\e}[1]{\ensuremath{\operatorname{e}^{#1}}}%DB
\newcommand  {\EcrossB}{\ensuremath{\vec{E}\times\vec{B}\;}}%DB
\newcommand  {\JcrossB}{\ensuremath{\vec{J}\times\vec{B}\;}}%DB
\newcommand  {\gradB}{\ensuremath{\nabla B\;}}%DB
\newcommand  {\infinity}{\ensuremath{\infty}}%DB
\newcommand  {\differential}[2]{\ensuremath{{\operatorname{d}\!{#1}\over\operatorname{d}\!{#2}}}}
\newcolumntype{x}[1]{>{\centering\hspace{0pt}}m{#1}}%DB
\newcolumntype{y}[1]{>{\raggedright\hspace{0pt}}m{#1}}%DB
\DeclareSIUnit{\atmosphere}{atm}%DB
\DeclareSIUnit{\amp}{A}%DB
\DeclareSIUnit{\torr}{torr}%DB
\DeclareSIUnit{\gauge}{AWG}%DB
\newcommand  {\Kocan}{Ko\v{c}an}%DB
\newcommand  {\inch}{\ensuremath{^{''}}}%DB
\newcommand  {\foot}{\ensuremath{^{'}}}%DB
\mathchardef\mhyphen="2D%DB good hyphen in math
\newcommand{\IV}{\ensuremath{I\mhyphen V}}%DB
\newcommand{\IVs}{\ensuremath{I\mhyphen V}s}%DB
\newcommand{\liningnums}[1]{{\fontfamily{pplx}\selectfont #1}}%DB
\newcommand{\CMod}{\mbox{C-Mod}}

\definecolor{myred}{rgb}{0.988,0.25,0.25}
\definecolor{RoyalBlue}{rgb}{0.988,0.25,0.25}

\makeatletter%DB
	\providecommand*{\diff}%
		{\@ifnextchar^{\DIfF}{\DIfF^{}}}
	\def\DIfF^#1{%
		\mathop{\mathrm{\mathstrut d}}%
			\nolimits^{#1}\gobblespace}
	\def\gobblespace{%
		\futurelet\diffarg\opspace}
	\def\opspace{%
		\let\DiffSpace\!%
		\ifx\diffarg(%
			\let\DiffSpace\relax
		\else
			\ifx\diffarg[%
				\let\DiffSpace\relax
			\else
				\ifx\diffarg\{%
					\let\DiffSpace\relax
				\fi\fi\fi\DiffSpace}
\newcommand*{\deriv}[3][]{%
	\frac{\diff^{#1}#2}{\diff #3^{#1}}}
\providecommand*{\pderiv}[3][]{%
	\frac{\partial^{#1}#2}%
		{\partial #3^{#1}}}
\makeatother

\newcommand*\myhrulefill{%
   \leavevmode\leaders\hrule depth-2pt height 2.4pt\hfill\kern0pt}

\newcommand\niceending[1]{%
  \begin{center}%
    \LARGE \myhrulefill \hspace{0.2cm} #1 \hspace{0.2cm} \myhrulefill%
  \end{center}}

\newcommand*\nicesectionending{\hfill\textcolor{myred}{$\bullet$}}
\newcommand*\nicechapterending{\hfill\textcolor{myred}{$\star$}}


% ****************************************************************************************************
% 4. Setup floats: tables, (sub)figures, and captions
% ****************************************************************************************************
\usepackage{tabularx} % better tables
	\setlength{\extrarowheight}{3pt} % increase table row height
\newcommand{\tableheadline}[1]{\multicolumn{1}{c}{\spacedlowsmallcaps{#1}}}
\newcommand{\myfloatalign}{\centering} % to be used with each float for alignment
%\usepackage{caption}
%\captionsetup{format=hang,font=small}
\usepackage[normal,bf,font=footnotesize,justification=raggedright]{caption}%DB
\usepackage{subfig}
% ****************************************************************************************************


% ****************************************************************************************************
% 5. Setup code listings
% ****************************************************************************************************
\usepackage{listings}
%\lstset{emph={trueIndex,root},emphstyle=\color{BlueViolet}}%\underbar} % for special keywords
\lstset{language=[LaTeX]Tex,%C++,
    keywordstyle=\color{RoyalBlue},%\bfseries,
    basicstyle=\small\ttfamily,
    %identifierstyle=\color{NavyBlue},
    commentstyle=\color{Green}\ttfamily,
    stringstyle=\rmfamily,
    numbers=none,%left,%
    numberstyle=\scriptsize,%\tiny
    stepnumber=5,
    numbersep=8pt,
    showstringspaces=false,
    breaklines=true,
    frameround=ftff,
    frame=single,
    belowcaptionskip=.75\baselineskip
    %frame=L
}
% ****************************************************************************************************    		 


% ****************************************************************************************************
% 6. PDFLaTeX, hyperreferences and citation backreferences
% ****************************************************************************************************
% ********************************************************************
% Using PDFLaTeX
% ********************************************************************
\PassOptionsToPackage{pdftex,hyperfootnotes=false,pdfpagelabels}{hyperref}
	\usepackage{hyperref}  % backref linktocpage pagebackref
\pdfcompresslevel=9
\pdfadjustspacing=1
\PassOptionsToPackage{pdftex}{graphicx}
	\usepackage{graphicx}

% ********************************************************************
% Setup the style of the backrefs from the bibliography
% (translate the options to any language you use)
% ********************************************************************
\newcommand{\backrefnotcitedstring}{\relax}%(Not cited.)
\newcommand{\backrefcitedsinglestring}[1]{(Cited on page~#1.)}
\newcommand{\backrefcitedmultistring}[1]{(Cited on pages~#1.)}
\ifthenelse{\boolean{enable-backrefs}}%
{%
		\PassOptionsToPackage{hyperpageref}{backref}
		\usepackage{backref} % to be loaded after hyperref package
		   \renewcommand{\backreftwosep}{ and~} % separate 2 pages
		   \renewcommand{\backreflastsep}{, and~} % separate last of longer list
		   \renewcommand*{\backref}[1]{}  % disable standard
		   \renewcommand*{\backrefalt}[4]{% detailed backref
		      \ifcase #1 %
		         \backrefnotcitedstring%
		      \or%
		         \backrefcitedsinglestring{#2}%
		      \else%
		         \backrefcitedmultistring{#2}%
		      \fi}%
}{\relax}

% ********************************************************************
% Hyperreferences
% ********************************************************************
\hypersetup{%
    %draft,	% = no hyperlinking at all (useful in b/w printouts)
    colorlinks=true, linktocpage=true, pdfstartpage=1, pdfstartview=FitV,%
    % uncomment the following line if you want to have black links (e.g., for printing)
    %colorlinks=false, linktocpage=false, pdfborder={0 0 0}, pdfstartpage=3, pdfstartview=FitV,%
    breaklinks=true, pdfpagemode=UseNone, pageanchor=true, pdfpagemode=UseOutlines,%
    plainpages=false, bookmarksnumbered, bookmarksopen=false, bookmarksopenlevel=1,%
    hypertexnames=true, pdfhighlight=/O,%nesting=true,%frenchlinks,%
    urlcolor=webbrown, linkcolor=RoyalBlue, citecolor=webgreen, %pagecolor=RoyalBlue,%
    %urlcolor=Black, linkcolor=Black, citecolor=Black, %pagecolor=Black,%
    pdftitle={\myTitle},%
    pdfauthor={\textcopyright\ \myName, \myUni, \myFaculty},%
    pdfsubject={},%
    pdfkeywords={},%
    pdfcreator={pdfLaTeX},%
    pdfproducer={LaTeX with hyperref and classicthesis}%
}

% ********************************************************************
% Setup autoreferences
% ********************************************************************
% There are some issues regarding autorefnames
% http://www.ureader.de/msg/136221647.aspx
% http://www.tex.ac.uk/cgi-bin/texfaq2html?label=latexwords
% you have to redefine the makros for the
% language you use, e.g., american, ngerman
% (as chosen when loading babel/AtBeginDocument)
% ********************************************************************
\makeatletter
\@ifpackageloaded{babel}%
    {%
       \addto\extrasamerican{%
					\renewcommand*{\figureautorefname}{Figure}%
					\renewcommand*{\tableautorefname}{Table}%
					\renewcommand*{\partautorefname}{Part}%
					\renewcommand*{\chapterautorefname}{Chapter}%
					\renewcommand*{\sectionautorefname}{Section}%
					\renewcommand*{\subsectionautorefname}{Section}%
					\renewcommand*{\subsubsectionautorefname}{Section}% 	
				}%
%        \addto\extrasngerman{%
% 					\renewcommand*{\paragraphautorefname}{Absatz}%
% 					\renewcommand*{\subparagraphautorefname}{Unterabsatz}%
% 					\renewcommand*{\footnoteautorefname}{Fu\"snote}%
% 					\renewcommand*{\FancyVerbLineautorefname}{Zeile}%
% 					\renewcommand*{\theoremautorefname}{Theorem}%
% 					\renewcommand*{\appendixautorefname}{Anhang}%
% 					\renewcommand*{\equationautorefname}{Gleichung}%
% 					\renewcommand*{\itemautorefname}{Punkt}%
% 				}%	
			% Fix to getting autorefs for subfigures right (thanks to Belinda Vogt for changing the definition)
			\providecommand{\subfigureautorefname}{\figureautorefname}%  			
    }{\relax}
\makeatother

% ****************************************************************************************************
% 7. Last calls before the bar closes
% ****************************************************************************************************
% ********************************************************************
% Development Stuff
% ********************************************************************
\listfiles
%\PassOptionsToPackage{l2tabu,orthodox,abort}{nag}
%	\usepackage{nag}
%\PassOptionsToPackage{warning, all}{onlyamsmath}
%	\usepackage{onlyamsmath}

% ********************************************************************
% Last, but not least...
% ********************************************************************
\usepackage{classicthesis}
% ****************************************************************************************************
\usepackage[capbesideposition={top}]{floatrow}   %allows adjustment of captions to fit figures %DB
\usepackage{wrapfig}    %allows figures to start in marge in protrude into text
\usepackage{calc}

\usepackage{changepage}
\newcommand{\pushtooutside}{%Checks if on odd or even page, pushes to align with outside margin of page
\strictpagecheck

\checkoddpage
\ifoddpage
  \hskip+2.0\marginspace
\else
  \hskip-2.0\marginspace
\fi
}

\newlength{\marginspace}
\setlength{\marginspace}{\marginparwidth+\marginparsep}
\setlength{\wrapoverhang}{\marginspace+\columnsep}

\makeatletter
\newcommand\margincaption[1]{%
  \par
  \setbox\@tempboxa=\hbox{\parbox{\marginparwidth}{\caption{#1}}}
  \@tempdima=\dimexpr\ht\@tempboxa+\dp\@tempboxa\relax
  \par\null\hfill\usebox\@tempboxa\par
  \vspace*{\dimexpr-\@tempdima-\baselineskip\relax}
}
\makeatother

\makeatletter
\newcommand*\smashcaption{%
        \def\FR@makecaption##1##2{%
                \vbox to\z@{%
                        \vss
                        \captionfont
                        {\captionlabelfont##1}\caption@lsep##2%
                        \par
                }%
        }%
        \caption
}
\makeatother

\setcounter{secnumdepth}{1}%only number sections %DB
\setcounter{tocdepth}{3}%include all levels in toc %DB

\usepackage[margincaption,outercaption,ragged,wide]{sidecap}

% rigth placed wrapfigure with right side caption
% wrapfigureScaption & wrapfigureRotScaption version 0.001
\newsavebox{\wImg}
\newsavebox{\wCap}
\newlength{\wImgH}
\newlength{\wImgW}
\newlength{\wCapH}
\newlength{\wCapW}
\newlength{\wImgTW}

\newcommand{\wrapfigureScaption}[3]{%
  \savebox{\wImg}{#1}
  \setlength{\wImgW}{\wd\wImg}
  \setlength{\wImgTW}{\wImgW}
  \setlength{\wCapW}{#3}
  \addtolength{\wImgTW}{\wCapW}
  \addtolength{\wImgTW}{10pt}
  \begin{wrapfigure}{r}{\wImgTW}
    \begin{minipage}[c]{\wImgW}
        \usebox{\wImg}
    \end{minipage}
    \hfill
    \begin{minipage}[c]{\wCapW}
        \caption{#2}
    \end{minipage}
  \end{wrapfigure}
}

%
% All the packages that I like, DB
%
\usepackage{amssymb}
\usepackage[page]{appendix}
\usepackage[loose]{units}
\usepackage[textwidth=2.5cm,textsize=small]{todonotes}
\usepackage{booktabs}
\usepackage{longtable}
\usepackage{pdflscape}
\usepackage{color}
\usepackage{chapterbib}
\usepackage{doi}


% Here's Gildea's Boilerplate Stuff.
% Copyright (c) 1987 by Stephen Gildea
% Permission to copy all or part of this work is granted, provided
% that the copies are not made or distributed for resale, and that
% the copyright notice and this notice are retained.

\makeatletter

\def\choosecase#1{#1}

%% Define all the pieces that go on the title page and the abstract.

% \title and \author already exist

\def\prevdegrees#1{\gdef\@prevdegrees{#1}}
\def\@prevdegrees{}

\def\department#1{\gdef\@department{#1}}

% If you are getting two degrees, use \and between the names.
\def\degree#1{\setbox0\hbox{#1}	 %for side effect of setting \@degreeword
  \gdef\@degree{#1}}

% \and is used inside the \degree argument to separate two degrees
\def\and{\gdef\@degreeword{degrees} \par and \par}
\def\@degreeword{degree}

% The copyright notice stuff is a tremendous mess.
%
% \@copyrightnotice is used by \maketitle to actually put text on the
% page; it defaults to ``Copyright MIT 19xx.  All rights reserved.''
% \copyrightnoticetext takes an argument and defined \@copyrightnotice
% to that argument.  \copyrightnotice takes an argument, and calls
% \copyrightnoticetext with that argument, preceeded by a copyright
% symbol and followed by ``All rights reserved.'' and the standard
% permission notice.
%
% If you use the 'vi' option, \copyrightnoticetext is used to set the
% copyright to ``(C) Your Name, Current Year in Roman Numerals.''
% followed by the permission notice.

% If there is no \copyrightnotice command, it is asssumed that MIT
% holds the copyright.  This commands adds the copyright symbol to the
% beginning, and puts the standard permission notice below.
%% ``All rights reserved'' added.  Krishna Sethuraman (1990)
\def\copyrightnotice#1{\copyrightnoticetext{\copyright\ #1.  All rights
reserved.\par\permission}}

% Occacionally you will need to exactly specify the text of the
% copyright notice.  The \copyrightnoticetext command is then useful.
\long\def\copyrightnoticetext#1{\gdef\@copyrightnotice{#1}}
\def\@copyrightnotice{\copyright\ \Mit\ \@degreeyear.  All rights reserved.}

%% `vi' documentclass option: Specifying this option automatically
%% copyrights the thesis to the author and gives MIT permission to copy and
%% distribute the document.  If you want, you can still specify
%% \copyrightnotice{stuff} to copyright to someone else, or
%% \copyrightnoticetext{stuff} to specify the exact text of the copyright
%% notice.
%%\ifodd\vithesis \copyrightnoticetext{\copyright\ \@author,
%%\uppercase\expandafter{\romannumeral\@degreeyear}.  All rights reserved.\par\permission}
%% or just
%%\@degreeyear}}
%%\typeout{Copyright given to author,
%%	permission to copy/distribute given to MIT.}
%%\else \typeout{Thesis document copyright MIT unless otherwise (manually) specified}
%%\fi

\def\thesisdate#1{\gdef\@thesisdate{#1}}

% typically just a month and year
\def\degreemonth#1{\gdef\@degreemonth{#1}}
\def\degreeyear#1{\gdef\@degreeyear{#1}}

% Usage: \supervisor{name}{title}
%        \chairman{name}{title}

% since there can be more than one supervisor,
% we build the appropriate boxes for the titlepage and
% the abstractpage as the user makes multiple calls
% to \supervisor
%\newbox\@titlesupervisor 	\newbox\@abstractsupervisor
%
%\def\supervisor#1#2{\setbox\@titlesupervisor\vbox
%  {\unvbox\@titlesupervisor \vskip 10pt% plus 1fil minus 1fil
%  \def\baselinestretch{1}\large
%  \signature{Certified by}{#1 \\ #2 \\ Thesis Supervisor}}
%  \setbox\@abstractsupervisor\vbox{\unvbox\@abstractsupervisor
%  \vskip\baselineskip \def\baselinestretch{1}\@normalsize
%  \par\noindent Thesis Supervisor: #1 \\ Title: #2}}

% supervisor
\def\supervisor#1#2{\gdef\@supervisorname{#1}\gdef\@supervisortitle{#2}}

% reader
\def\reader#1#2{\gdef\@readername{#1}\gdef\@readertitle{#2}}

% department chairman, not thesis committee chairman
\def\chairman#1#2{\gdef\@chairmanname{#1}\gdef\@chairmantitle{#2}}

%% `upcase' documentclass option: \choosecase is defined either as a dummy or
%% a macro to change the (expanded) argument to uppercase.
\def\maketitle{\begin{titlepage}
\large
{\def\baselinestretch{1.2}\Large\bf \choosecase{\@title} \par}
by\par
{\Large  \choosecase{\@author}}
\par
\@prevdegrees
\par
\choosecase{Submitted to the} \choosecase{\@department} \\
\choosecase{in partial fulfillment of the requirements for the}
\choosecase{\@degreeword}
\choosecase{of}
\par
\choosecase{\@degree}
\par
at the
\par\MIT\par
\@degreemonth\ \@degreeyear
\par
\@copyrightnotice
\par
\vskip 3\baselineskip
\signature{Author}{\@department \\ \@thesisdate}
\par
\vfill
\signature{Certified by}{\@supervisorname \\ \@supervisortitle \\ Thesis Supervisor}
\par
\vfill
\signature{Certified by}{\@readername \\ \@readertitle \\ Thesis Reader}
\par
\vfill
\signature{Accepted by}{\@chairmanname \\ \@chairmantitle}
\vfill
\end{titlepage}}

% this environment should probably be called abstract,
% but we want people to also be able to get at the more
% basic abstract environment
\def\abstractpage{\cleardoublepage
\begin{center}{\large{\bf \@title} \\
by \\
\@author \\[\baselineskip]}
\par
\def\baselinestretch{1}\@normalsize
Submitted to the \@department \\
on \@thesisdate, in partial fulfillment of the \\
requirements for the \@degreeword\ of \\
\@degree
\end{center}
\par
\begin{abstract}}

%% Changed from \unvbox to \unvcopy for use with multiple copies of abstract
%% page.
%% Krishna Sethuraman (1990)
\def\endabstractpage{\end{abstract}\noindent
    \vskip\baselineskip \def\baselinestretch{1}\@normalsize
    \par\noindent Thesis Supervisor: \@supervisorname \\ \@supervisortitle
    \vskip\baselineskip \def\baselinestretch{1}\@normalsize
    \par\noindent Thesis Reader: \@readername \\ \@readertitle
    \newpage
    }

%% This counter is used to save the page number for the second copy of
%% the abstract.
\newcounter{savepage}

% You can use the titlepage environment to do it all yourself if you
% don't want to use \maketitle.  If the titlepage environment, the
% paragraph skip is infinitely stretchable, so if you leave a blank line
% between lines that you want space between, the space will stretch so
% that the title page fills up the entire page.
\def\titlepage{\cleardoublepage\centering
  \thispagestyle{empty}
  \parindent 0pt \parskip 10pt plus 1fil minus 1fil
  \def\baselinestretch{1}\@normalsize\vbox to \vsize\bgroup\vbox to 9in\bgroup}
% The \kern0pt pushes any depth into the height.  Thanks to Richard Stone.
\def\endtitlepage{\par\kern 0pt\egroup\vss\egroup\newpage}

\def\MIT{MASSACHUSETTS INSTITUTE OF TECHNOLOGY}
\def\Mit{Massachusetts Institute of Technology}

\def\permission{\par\noindent{\centering
   The author hereby grants to MIT permission to reproduce and
   distribute publicly paper and electronic copies of this thesis
   document in whole or in part.}\par}

\def\signature#1#2{\par\noindent#1\dotfill\null\\*
  {\raggedleft #2\par}}

\def\abstract{\subsection*{Abstract}\small\def\baselinestretch{1}\@normalsize}
\def\endabstract{\par}

\makeatother

% ****************************************************************************************************
% 8. Further adjustments (experimental)
% ****************************************************************************************************
% ********************************************************************
% Changing the text area
% ********************************************************************
%\linespread{1.05} % a bit more for Palatino
%\areaset[current]{312pt}{761pt} % 686 (factor 2.2) + 33 head + 42 head \the\footskip
%\setlength{\marginparwidth}{7em}%
%\setlength{\marginparsep}{2em}%

% ********************************************************************
% Using different fonts
% ********************************************************************
%\usepackage[oldstylenums]{kpfonts} % oldstyle notextcomp
%\usepackage[osf]{libertine}
%\usepackage{hfoldsty} % Computer Modern with osf
%\usepackage[light,condensed,math]{iwona}
%\renewcommand{\sfdefault}{iwona}
%\usepackage{lmodern} % <-- no osf support :-(
%\usepackage[urw-garamond]{mathdesign} <-- no osf support :-(
% ****************************************************************************************************


% ****************************************************************************************************
% If you like the classicthesis, then I would appreciate a postcard.
% My address can be found in the file ClassicThesis.pdf. A collection
% of the postcards I received so far is available online at
% http://postcards.miede.de
% ****************************************************************************************************

% ****************************************************************************************************
% 1. Configure classicthesis for your needs here, e.g., remove "drafting" below
% in order to deactivate the time-stamp on the pages
% ****************************************************************************************************
%\PassOptionsToPackage{eulerchapternumbers,%drafting,%
%				 pdfspacing,%floatperchapter,%linedheaders,%
%				 subfig,beramono,eulermath}{classicthesis}		

\PassOptionsToPackage{eulerchapternumbers,drafting,%
				 pdfspacing,%floatperchapter,%linedheaders,%
				 subfig,eulermath}{classicthesis}									
% ********************************************************************
% Available options for classicthesis.sty
% (see ClassicThesis.pdf for more information):
% drafting
% parts nochapters linedheaders
% eulerchapternumbers beramono eulermath pdfspacing minionprospacing
% tocaligned dottedtoc manychapters
% listings floatperchapter subfig
% ********************************************************************

% ********************************************************************
% Triggers for this config
% ********************************************************************
\usepackage{ifthen}
\newboolean{enable-backrefs} % enable backrefs in the bibliography
\setboolean{enable-backrefs}{false} % true false
% ****************************************************************************************************


\usepackage{geometry}%[showframe] allows differnt margins on title page, DB

% ****************************************************************************************************
% 2. Personal data and user ad-hoc commands
% ****************************************************************************************************
\newcommand{\myTitle}{A Classic Thesis Style\xspace}
\newcommand{\mySubtitle}{An Homage to The Elements of Typographic Style\xspace}
\newcommand{\myDegree}{Doktor-Ingenieur (Dr.-Ing.)\xspace}
\newcommand{\myName}{Andr\'e Miede\xspace}
\newcommand{\myProf}{Put name here\xspace}
\newcommand{\myOtherProf}{Put name here\xspace}
\newcommand{\mySupervisor}{Put name here\xspace}
\newcommand{\myFaculty}{Put data here\xspace}
\newcommand{\myDepartment}{Put data here\xspace}
\newcommand{\myUni}{Put data here\xspace}
\newcommand{\myLocation}{Darmstadt\xspace}
\newcommand{\myTime}{August 2012\xspace}
\newcommand{\myVersion}{version 4.1\xspace}

% ********************************************************************
% Setup, finetuning, and useful commands
% ********************************************************************
\newcounter{dummy} % necessary for correct hyperlinks (to index, bib, etc.)
\newlength{\abcd} % for ab..z string length calculation
\providecommand{\mLyX}{L\kern-.1667em\lower.25em\hbox{Y}\kern-.125emX\@}
\newcommand{\ie}{i.\,e.,\;}
\newcommand{\Ie}{I.\,e.,\;}
\newcommand{\eg}{e.\,g.,\;}
\newcommand{\Eg}{E.\,g.,\;}
% ****************************************************************************************************


% ****************************************************************************************************
% 3. Loading some handy packages
% ****************************************************************************************************
% ********************************************************************
% Packages with options that might require adjustments
% ********************************************************************
% \PassOptionsToPackage{utf8}{inputenc}	% latin9 (ISO-8859-9) = latin1+"Euro sign"
%  \usepackage{inputenc}				

% \PassOptionsToPackage{russian,american}{babel}   % change this to your language(s)
% % Spanish languages need extra options in order to work with this template
% %\PassOptionsToPackage{spanish,es-lcroman}{babel}
%  \usepackage{babel}					

\PassOptionsToPackage{square,numbers,sort}{natbib}
 \usepackage{natbib}				

\PassOptionsToPackage{fleqn}{amsmath}		% math environments and more by the AMS
 \usepackage{amsmath}

% ********************************************************************
% General useful packages
% ********************************************************************
\PassOptionsToPackage{T1}{fontenc} % T2A for cyrillics
	\usepackage{fontenc}
\PassOptionsToPackage{utf8}{inputenc}	% latin9 (ISO-8859-9) = latin1+"Euro sign"
	\usepackage{inputenc}	
\usepackage{textcomp} % fix warning with missing font shapes
\usepackage{scrhack} % fix warnings when using KOMA with listings package
\usepackage{xspace} % to get the spacing after macros right
\usepackage{mparhack} % get marginpar right
\usepackage{fixltx2e} % fixes some LaTeX stuff
\usepackage[detect-all]{siunitx}%convenient units, DB
\sisetup{mode=text,range-phrase = {\text{~to~}}}%allows ranges to be used in math%DB
\usepackage{setspace}%allows double spacing for draft, DB
\usepackage{bm}%allows bold math, DB
%\usepackage{mcaption}%allows captions in margins, DB
\PassOptionsToPackage{printonlyused,smaller}{acronym}
	\usepackage{acronym} % nice macros for handling all acronyms in the thesis
%\renewcommand*{\acsfont}[1]{\textssc{#1}} % for MinionPro
\renewcommand{\bflabel}[1]{{#1}\hfill} % fix the list of acronyms
\usepackage{xcolor}

\PassOptionsToPackage{russian,american}{babel}   % change this to your language(s)
% Spanish languages need extra options in order to work with this template
%\PassOptionsToPackage{spanish,es-lcroman}{babel}
 \usepackage{babel}
% ****************************************************************************************************
%
% Some new commands that I like, DB
%
\renewcommand{\sup}[1]{\ensuremath{^{\mathrm{#1}}}}%DB
\newcommand  {\sub}[1]{\ensuremath{_{\mathrm{#1}}}}%DB
\newcommand  {\e}[1]{\ensuremath{\operatorname{e}^{#1}}}%DB
\newcommand  {\EcrossB}{\ensuremath{\vec{E}\times\vec{B}\;}}%DB
\newcommand  {\JcrossB}{\ensuremath{\vec{J}\times\vec{B}\;}}%DB
\newcommand  {\gradB}{\ensuremath{\nabla B\;}}%DB
\newcommand  {\infinity}{\ensuremath{\infty}}%DB
\newcommand  {\differential}[2]{\ensuremath{{\operatorname{d}\!{#1}\over\operatorname{d}\!{#2}}}}
\newcolumntype{x}[1]{>{\centering\hspace{0pt}}m{#1}}%DB
\newcolumntype{y}[1]{>{\raggedright\hspace{0pt}}m{#1}}%DB
\DeclareSIUnit{\atmosphere}{atm}%DB
\DeclareSIUnit{\amp}{A}%DB
\DeclareSIUnit{\torr}{torr}%DB
\DeclareSIUnit{\gauge}{AWG}%DB
\newcommand  {\Kocan}{Ko\v{c}an}%DB
\newcommand  {\inch}{\ensuremath{^{''}}}%DB
\newcommand  {\foot}{\ensuremath{^{'}}}%DB
\mathchardef\mhyphen="2D%DB good hyphen in math
\newcommand{\IV}{\ensuremath{I\mhyphen V}}%DB
\newcommand{\IVs}{\ensuremath{I\mhyphen V}s}%DB
\newcommand{\liningnums}[1]{{\fontfamily{pplx}\selectfont #1}}%DB
\newcommand{\CMod}{\mbox{C-Mod}}

\definecolor{myred}{rgb}{0.988,0.25,0.25}
\definecolor{RoyalBlue}{rgb}{0.988,0.25,0.25}

\makeatletter%DB
	\providecommand*{\diff}%
		{\@ifnextchar^{\DIfF}{\DIfF^{}}}
	\def\DIfF^#1{%
		\mathop{\mathrm{\mathstrut d}}%
			\nolimits^{#1}\gobblespace}
	\def\gobblespace{%
		\futurelet\diffarg\opspace}
	\def\opspace{%
		\let\DiffSpace\!%
		\ifx\diffarg(%
			\let\DiffSpace\relax
		\else
			\ifx\diffarg[%
				\let\DiffSpace\relax
			\else
				\ifx\diffarg\{%
					\let\DiffSpace\relax
				\fi\fi\fi\DiffSpace}
\newcommand*{\deriv}[3][]{%
	\frac{\diff^{#1}#2}{\diff #3^{#1}}}
\providecommand*{\pderiv}[3][]{%
	\frac{\partial^{#1}#2}%
		{\partial #3^{#1}}}
\makeatother

\newcommand*\myhrulefill{%
   \leavevmode\leaders\hrule depth-2pt height 2.4pt\hfill\kern0pt}

\newcommand\niceending[1]{%
  \begin{center}%
    \LARGE \myhrulefill \hspace{0.2cm} #1 \hspace{0.2cm} \myhrulefill%
  \end{center}}

\newcommand*\nicesectionending{\hfill\textcolor{myred}{$\bullet$}}
\newcommand*\nicechapterending{\hfill\textcolor{myred}{$\star$}}


% ****************************************************************************************************
% 4. Setup floats: tables, (sub)figures, and captions
% ****************************************************************************************************
\usepackage{tabularx} % better tables
	\setlength{\extrarowheight}{3pt} % increase table row height
\newcommand{\tableheadline}[1]{\multicolumn{1}{c}{\spacedlowsmallcaps{#1}}}
\newcommand{\myfloatalign}{\centering} % to be used with each float for alignment
%\usepackage{caption}
%\captionsetup{format=hang,font=small}
\usepackage[normal,bf,font=footnotesize,justification=raggedright]{caption}%DB
\usepackage{subfig}
% ****************************************************************************************************


% ****************************************************************************************************
% 5. Setup code listings
% ****************************************************************************************************
\usepackage{listings}
%\lstset{emph={trueIndex,root},emphstyle=\color{BlueViolet}}%\underbar} % for special keywords
\lstset{language=[LaTeX]Tex,%C++,
    keywordstyle=\color{RoyalBlue},%\bfseries,
    basicstyle=\small\ttfamily,
    %identifierstyle=\color{NavyBlue},
    commentstyle=\color{Green}\ttfamily,
    stringstyle=\rmfamily,
    numbers=none,%left,%
    numberstyle=\scriptsize,%\tiny
    stepnumber=5,
    numbersep=8pt,
    showstringspaces=false,
    breaklines=true,
    frameround=ftff,
    frame=single,
    belowcaptionskip=.75\baselineskip
    %frame=L
}
% ****************************************************************************************************    		 


% ****************************************************************************************************
% 6. PDFLaTeX, hyperreferences and citation backreferences
% ****************************************************************************************************
% ********************************************************************
% Using PDFLaTeX
% ********************************************************************
\PassOptionsToPackage{pdftex,hyperfootnotes=false,pdfpagelabels}{hyperref}
	\usepackage{hyperref}  % backref linktocpage pagebackref
\pdfcompresslevel=9
\pdfadjustspacing=1
\PassOptionsToPackage{pdftex}{graphicx}
	\usepackage{graphicx}

% ********************************************************************
% Setup the style of the backrefs from the bibliography
% (translate the options to any language you use)
% ********************************************************************
\newcommand{\backrefnotcitedstring}{\relax}%(Not cited.)
\newcommand{\backrefcitedsinglestring}[1]{(Cited on page~#1.)}
\newcommand{\backrefcitedmultistring}[1]{(Cited on pages~#1.)}
\ifthenelse{\boolean{enable-backrefs}}%
{%
		\PassOptionsToPackage{hyperpageref}{backref}
		\usepackage{backref} % to be loaded after hyperref package
		   \renewcommand{\backreftwosep}{ and~} % separate 2 pages
		   \renewcommand{\backreflastsep}{, and~} % separate last of longer list
		   \renewcommand*{\backref}[1]{}  % disable standard
		   \renewcommand*{\backrefalt}[4]{% detailed backref
		      \ifcase #1 %
		         \backrefnotcitedstring%
		      \or%
		         \backrefcitedsinglestring{#2}%
		      \else%
		         \backrefcitedmultistring{#2}%
		      \fi}%
}{\relax}

% ********************************************************************
% Hyperreferences
% ********************************************************************
\hypersetup{%
    %draft,	% = no hyperlinking at all (useful in b/w printouts)
    colorlinks=true, linktocpage=true, pdfstartpage=1, pdfstartview=FitV,%
    % uncomment the following line if you want to have black links (e.g., for printing)
    %colorlinks=false, linktocpage=false, pdfborder={0 0 0}, pdfstartpage=3, pdfstartview=FitV,%
    breaklinks=true, pdfpagemode=UseNone, pageanchor=true, pdfpagemode=UseOutlines,%
    plainpages=false, bookmarksnumbered, bookmarksopen=false, bookmarksopenlevel=1,%
    hypertexnames=true, pdfhighlight=/O,%nesting=true,%frenchlinks,%
    urlcolor=webbrown, linkcolor=RoyalBlue, citecolor=webgreen, %pagecolor=RoyalBlue,%
    %urlcolor=Black, linkcolor=Black, citecolor=Black, %pagecolor=Black,%
    pdftitle={\myTitle},%
    pdfauthor={\textcopyright\ \myName, \myUni, \myFaculty},%
    pdfsubject={},%
    pdfkeywords={},%
    pdfcreator={pdfLaTeX},%
    pdfproducer={LaTeX with hyperref and classicthesis}%
}

% ********************************************************************
% Setup autoreferences
% ********************************************************************
% There are some issues regarding autorefnames
% http://www.ureader.de/msg/136221647.aspx
% http://www.tex.ac.uk/cgi-bin/texfaq2html?label=latexwords
% you have to redefine the makros for the
% language you use, e.g., american, ngerman
% (as chosen when loading babel/AtBeginDocument)
% ********************************************************************
\makeatletter
\@ifpackageloaded{babel}%
    {%
       \addto\extrasamerican{%
					\renewcommand*{\figureautorefname}{Figure}%
					\renewcommand*{\tableautorefname}{Table}%
					\renewcommand*{\partautorefname}{Part}%
					\renewcommand*{\chapterautorefname}{Chapter}%
					\renewcommand*{\sectionautorefname}{Section}%
					\renewcommand*{\subsectionautorefname}{Section}%
					\renewcommand*{\subsubsectionautorefname}{Section}% 	
				}%
%        \addto\extrasngerman{%
% 					\renewcommand*{\paragraphautorefname}{Absatz}%
% 					\renewcommand*{\subparagraphautorefname}{Unterabsatz}%
% 					\renewcommand*{\footnoteautorefname}{Fu\"snote}%
% 					\renewcommand*{\FancyVerbLineautorefname}{Zeile}%
% 					\renewcommand*{\theoremautorefname}{Theorem}%
% 					\renewcommand*{\appendixautorefname}{Anhang}%
% 					\renewcommand*{\equationautorefname}{Gleichung}%
% 					\renewcommand*{\itemautorefname}{Punkt}%
% 				}%	
			% Fix to getting autorefs for subfigures right (thanks to Belinda Vogt for changing the definition)
			\providecommand{\subfigureautorefname}{\figureautorefname}%  			
    }{\relax}
\makeatother

% ****************************************************************************************************
% 7. Last calls before the bar closes
% ****************************************************************************************************
% ********************************************************************
% Development Stuff
% ********************************************************************
\listfiles
%\PassOptionsToPackage{l2tabu,orthodox,abort}{nag}
%	\usepackage{nag}
%\PassOptionsToPackage{warning, all}{onlyamsmath}
%	\usepackage{onlyamsmath}

% ********************************************************************
% Last, but not least...
% ********************************************************************
\usepackage{classicthesis}
% ****************************************************************************************************
\usepackage[capbesideposition={top}]{floatrow}   %allows adjustment of captions to fit figures %DB
\usepackage{wrapfig}    %allows figures to start in marge in protrude into text
\usepackage{calc}

\usepackage{changepage}
\newcommand{\pushtooutside}{%Checks if on odd or even page, pushes to align with outside margin of page
\strictpagecheck

\checkoddpage
\ifoddpage
  \hskip+2.0\marginspace
\else
  \hskip-2.0\marginspace
\fi
}

\newlength{\marginspace}
\setlength{\marginspace}{\marginparwidth+\marginparsep}
\setlength{\wrapoverhang}{\marginspace+\columnsep}

\makeatletter
\newcommand\margincaption[1]{%
  \par
  \setbox\@tempboxa=\hbox{\parbox{\marginparwidth}{\caption{#1}}}
  \@tempdima=\dimexpr\ht\@tempboxa+\dp\@tempboxa\relax
  \par\null\hfill\usebox\@tempboxa\par
  \vspace*{\dimexpr-\@tempdima-\baselineskip\relax}
}
\makeatother

\makeatletter
\newcommand*\smashcaption{%
        \def\FR@makecaption##1##2{%
                \vbox to\z@{%
                        \vss
                        \captionfont
                        {\captionlabelfont##1}\caption@lsep##2%
                        \par
                }%
        }%
        \caption
}
\makeatother

\setcounter{secnumdepth}{1}%only number sections %DB
\setcounter{tocdepth}{3}%include all levels in toc %DB

\usepackage[margincaption,outercaption,ragged,wide]{sidecap}

% rigth placed wrapfigure with right side caption
% wrapfigureScaption & wrapfigureRotScaption version 0.001
\newsavebox{\wImg}
\newsavebox{\wCap}
\newlength{\wImgH}
\newlength{\wImgW}
\newlength{\wCapH}
\newlength{\wCapW}
\newlength{\wImgTW}

\newcommand{\wrapfigureScaption}[3]{%
  \savebox{\wImg}{#1}
  \setlength{\wImgW}{\wd\wImg}
  \setlength{\wImgTW}{\wImgW}
  \setlength{\wCapW}{#3}
  \addtolength{\wImgTW}{\wCapW}
  \addtolength{\wImgTW}{10pt}
  \begin{wrapfigure}{r}{\wImgTW}
    \begin{minipage}[c]{\wImgW}
        \usebox{\wImg}
    \end{minipage}
    \hfill
    \begin{minipage}[c]{\wCapW}
        \caption{#2}
    \end{minipage}
  \end{wrapfigure}
}

%
% All the packages that I like, DB
%
\usepackage{amssymb}
\usepackage[page]{appendix}
\usepackage[loose]{units}
\usepackage[textwidth=2.5cm,textsize=small]{todonotes}
\usepackage{booktabs}
\usepackage{longtable}
\usepackage{pdflscape}
\usepackage{color}
\usepackage{chapterbib}
\usepackage{doi}


% Here's Gildea's Boilerplate Stuff.
% Copyright (c) 1987 by Stephen Gildea
% Permission to copy all or part of this work is granted, provided
% that the copies are not made or distributed for resale, and that
% the copyright notice and this notice are retained.

\makeatletter

\def\choosecase#1{#1}

%% Define all the pieces that go on the title page and the abstract.

% \title and \author already exist

\def\prevdegrees#1{\gdef\@prevdegrees{#1}}
\def\@prevdegrees{}

\def\department#1{\gdef\@department{#1}}

% If you are getting two degrees, use \and between the names.
\def\degree#1{\setbox0\hbox{#1}	 %for side effect of setting \@degreeword
  \gdef\@degree{#1}}

% \and is used inside the \degree argument to separate two degrees
\def\and{\gdef\@degreeword{degrees} \par and \par}
\def\@degreeword{degree}

% The copyright notice stuff is a tremendous mess.
%
% \@copyrightnotice is used by \maketitle to actually put text on the
% page; it defaults to ``Copyright MIT 19xx.  All rights reserved.''
% \copyrightnoticetext takes an argument and defined \@copyrightnotice
% to that argument.  \copyrightnotice takes an argument, and calls
% \copyrightnoticetext with that argument, preceeded by a copyright
% symbol and followed by ``All rights reserved.'' and the standard
% permission notice.
%
% If you use the 'vi' option, \copyrightnoticetext is used to set the
% copyright to ``(C) Your Name, Current Year in Roman Numerals.''
% followed by the permission notice.

% If there is no \copyrightnotice command, it is asssumed that MIT
% holds the copyright.  This commands adds the copyright symbol to the
% beginning, and puts the standard permission notice below.
%% ``All rights reserved'' added.  Krishna Sethuraman (1990)
\def\copyrightnotice#1{\copyrightnoticetext{\copyright\ #1.  All rights
reserved.\par\permission}}

% Occacionally you will need to exactly specify the text of the
% copyright notice.  The \copyrightnoticetext command is then useful.
\long\def\copyrightnoticetext#1{\gdef\@copyrightnotice{#1}}
\def\@copyrightnotice{\copyright\ \Mit\ \@degreeyear.  All rights reserved.}

%% `vi' documentclass option: Specifying this option automatically
%% copyrights the thesis to the author and gives MIT permission to copy and
%% distribute the document.  If you want, you can still specify
%% \copyrightnotice{stuff} to copyright to someone else, or
%% \copyrightnoticetext{stuff} to specify the exact text of the copyright
%% notice.
%%\ifodd\vithesis \copyrightnoticetext{\copyright\ \@author,
%%\uppercase\expandafter{\romannumeral\@degreeyear}.  All rights reserved.\par\permission}
%% or just
%%\@degreeyear}}
%%\typeout{Copyright given to author,
%%	permission to copy/distribute given to MIT.}
%%\else \typeout{Thesis document copyright MIT unless otherwise (manually) specified}
%%\fi

\def\thesisdate#1{\gdef\@thesisdate{#1}}

% typically just a month and year
\def\degreemonth#1{\gdef\@degreemonth{#1}}
\def\degreeyear#1{\gdef\@degreeyear{#1}}

% Usage: \supervisor{name}{title}
%        \chairman{name}{title}

% since there can be more than one supervisor,
% we build the appropriate boxes for the titlepage and
% the abstractpage as the user makes multiple calls
% to \supervisor
%\newbox\@titlesupervisor 	\newbox\@abstractsupervisor
%
%\def\supervisor#1#2{\setbox\@titlesupervisor\vbox
%  {\unvbox\@titlesupervisor \vskip 10pt% plus 1fil minus 1fil
%  \def\baselinestretch{1}\large
%  \signature{Certified by}{#1 \\ #2 \\ Thesis Supervisor}}
%  \setbox\@abstractsupervisor\vbox{\unvbox\@abstractsupervisor
%  \vskip\baselineskip \def\baselinestretch{1}\@normalsize
%  \par\noindent Thesis Supervisor: #1 \\ Title: #2}}

% supervisor
\def\supervisor#1#2{\gdef\@supervisorname{#1}\gdef\@supervisortitle{#2}}

% reader
\def\reader#1#2{\gdef\@readername{#1}\gdef\@readertitle{#2}}

% department chairman, not thesis committee chairman
\def\chairman#1#2{\gdef\@chairmanname{#1}\gdef\@chairmantitle{#2}}

%% `upcase' documentclass option: \choosecase is defined either as a dummy or
%% a macro to change the (expanded) argument to uppercase.
\def\maketitle{\begin{titlepage}
\large
{\def\baselinestretch{1.2}\Large\bf \choosecase{\@title} \par}
by\par
{\Large  \choosecase{\@author}}
\par
\@prevdegrees
\par
\choosecase{Submitted to the} \choosecase{\@department} \\
\choosecase{in partial fulfillment of the requirements for the}
\choosecase{\@degreeword}
\choosecase{of}
\par
\choosecase{\@degree}
\par
at the
\par\MIT\par
\@degreemonth\ \@degreeyear
\par
\@copyrightnotice
\par
\vskip 3\baselineskip
\signature{Author}{\@department \\ \@thesisdate}
\par
\vfill
\signature{Certified by}{\@supervisorname \\ \@supervisortitle \\ Thesis Supervisor}
\par
\vfill
\signature{Certified by}{\@readername \\ \@readertitle \\ Thesis Reader}
\par
\vfill
\signature{Accepted by}{\@chairmanname \\ \@chairmantitle}
\vfill
\end{titlepage}}

% this environment should probably be called abstract,
% but we want people to also be able to get at the more
% basic abstract environment
\def\abstractpage{\cleardoublepage
\begin{center}{\large{\bf \@title} \\
by \\
\@author \\[\baselineskip]}
\par
\def\baselinestretch{1}\@normalsize
Submitted to the \@department \\
on \@thesisdate, in partial fulfillment of the \\
requirements for the \@degreeword\ of \\
\@degree
\end{center}
\par
\begin{abstract}}

%% Changed from \unvbox to \unvcopy for use with multiple copies of abstract
%% page.
%% Krishna Sethuraman (1990)
\def\endabstractpage{\end{abstract}\noindent
    \vskip\baselineskip \def\baselinestretch{1}\@normalsize
    \par\noindent Thesis Supervisor: \@supervisorname \\ \@supervisortitle
    \vskip\baselineskip \def\baselinestretch{1}\@normalsize
    \par\noindent Thesis Reader: \@readername \\ \@readertitle
    \newpage
    }

%% This counter is used to save the page number for the second copy of
%% the abstract.
\newcounter{savepage}

% You can use the titlepage environment to do it all yourself if you
% don't want to use \maketitle.  If the titlepage environment, the
% paragraph skip is infinitely stretchable, so if you leave a blank line
% between lines that you want space between, the space will stretch so
% that the title page fills up the entire page.
\def\titlepage{\cleardoublepage\centering
  \thispagestyle{empty}
  \parindent 0pt \parskip 10pt plus 1fil minus 1fil
  \def\baselinestretch{1}\@normalsize\vbox to \vsize\bgroup\vbox to 9in\bgroup}
% The \kern0pt pushes any depth into the height.  Thanks to Richard Stone.
\def\endtitlepage{\par\kern 0pt\egroup\vss\egroup\newpage}

\def\MIT{MASSACHUSETTS INSTITUTE OF TECHNOLOGY}
\def\Mit{Massachusetts Institute of Technology}

\def\permission{\par\noindent{\centering
   The author hereby grants to MIT permission to reproduce and
   distribute publicly paper and electronic copies of this thesis
   document in whole or in part.}\par}

\def\signature#1#2{\par\noindent#1\dotfill\null\\*
  {\raggedleft #2\par}}

\def\abstract{\subsection*{Abstract}\small\def\baselinestretch{1}\@normalsize}
\def\endabstract{\par}

\makeatother

% ****************************************************************************************************
% 8. Further adjustments (experimental)
% ****************************************************************************************************
% ********************************************************************
% Changing the text area
% ********************************************************************
%\linespread{1.05} % a bit more for Palatino
%\areaset[current]{312pt}{761pt} % 686 (factor 2.2) + 33 head + 42 head \the\footskip
%\setlength{\marginparwidth}{7em}%
%\setlength{\marginparsep}{2em}%

% ********************************************************************
% Using different fonts
% ********************************************************************
%\usepackage[oldstylenums]{kpfonts} % oldstyle notextcomp
%\usepackage[osf]{libertine}
%\usepackage{hfoldsty} % Computer Modern with osf
%\usepackage[light,condensed,math]{iwona}
%\renewcommand{\sfdefault}{iwona}
%\usepackage{lmodern} % <-- no osf support :-(
%\usepackage[urw-garamond]{mathdesign} <-- no osf support :-(
% ****************************************************************************************************


% ****************************************************************************************************
% If you like the classicthesis, then I would appreciate a postcard.
% My address can be found in the file ClassicThesis.pdf. A collection
% of the postcards I received so far is available online at
% http://postcards.miede.de
% ****************************************************************************************************

% ****************************************************************************************************
% 1. Configure classicthesis for your needs here, e.g., remove "drafting" below
% in order to deactivate the time-stamp on the pages
% ****************************************************************************************************
%\PassOptionsToPackage{eulerchapternumbers,%drafting,%
%				 pdfspacing,%floatperchapter,%linedheaders,%
%				 subfig,beramono,eulermath}{classicthesis}		

\PassOptionsToPackage{eulerchapternumbers,drafting,%
				 pdfspacing,%floatperchapter,%linedheaders,%
				 subfig,eulermath}{classicthesis}									
% ********************************************************************
% Available options for classicthesis.sty
% (see ClassicThesis.pdf for more information):
% drafting
% parts nochapters linedheaders
% eulerchapternumbers beramono eulermath pdfspacing minionprospacing
% tocaligned dottedtoc manychapters
% listings floatperchapter subfig
% ********************************************************************

% ********************************************************************
% Triggers for this config
% ********************************************************************
\usepackage{ifthen}
\newboolean{enable-backrefs} % enable backrefs in the bibliography
\setboolean{enable-backrefs}{false} % true false
% ****************************************************************************************************


\usepackage{geometry}%[showframe] allows differnt margins on title page, DB

% ****************************************************************************************************
% 2. Personal data and user ad-hoc commands
% ****************************************************************************************************
\newcommand{\myTitle}{A Classic Thesis Style\xspace}
\newcommand{\mySubtitle}{An Homage to The Elements of Typographic Style\xspace}
\newcommand{\myDegree}{Doktor-Ingenieur (Dr.-Ing.)\xspace}
\newcommand{\myName}{Andr\'e Miede\xspace}
\newcommand{\myProf}{Put name here\xspace}
\newcommand{\myOtherProf}{Put name here\xspace}
\newcommand{\mySupervisor}{Put name here\xspace}
\newcommand{\myFaculty}{Put data here\xspace}
\newcommand{\myDepartment}{Put data here\xspace}
\newcommand{\myUni}{Put data here\xspace}
\newcommand{\myLocation}{Darmstadt\xspace}
\newcommand{\myTime}{August 2012\xspace}
\newcommand{\myVersion}{version 4.1\xspace}

% ********************************************************************
% Setup, finetuning, and useful commands
% ********************************************************************
\newcounter{dummy} % necessary for correct hyperlinks (to index, bib, etc.)
\newlength{\abcd} % for ab..z string length calculation
\providecommand{\mLyX}{L\kern-.1667em\lower.25em\hbox{Y}\kern-.125emX\@}
\newcommand{\ie}{i.\,e.,\;}
\newcommand{\Ie}{I.\,e.,\;}
\newcommand{\eg}{e.\,g.,\;}
\newcommand{\Eg}{E.\,g.,\;}
% ****************************************************************************************************


% ****************************************************************************************************
% 3. Loading some handy packages
% ****************************************************************************************************
% ********************************************************************
% Packages with options that might require adjustments
% ********************************************************************
% \PassOptionsToPackage{utf8}{inputenc}	% latin9 (ISO-8859-9) = latin1+"Euro sign"
%  \usepackage{inputenc}				

% \PassOptionsToPackage{russian,american}{babel}   % change this to your language(s)
% % Spanish languages need extra options in order to work with this template
% %\PassOptionsToPackage{spanish,es-lcroman}{babel}
%  \usepackage{babel}					

\PassOptionsToPackage{square,numbers,sort}{natbib}
 \usepackage{natbib}				

\PassOptionsToPackage{fleqn}{amsmath}		% math environments and more by the AMS
 \usepackage{amsmath}

% ********************************************************************
% General useful packages
% ********************************************************************
\PassOptionsToPackage{T1}{fontenc} % T2A for cyrillics
	\usepackage{fontenc}
\PassOptionsToPackage{utf8}{inputenc}	% latin9 (ISO-8859-9) = latin1+"Euro sign"
	\usepackage{inputenc}	
\usepackage{textcomp} % fix warning with missing font shapes
\usepackage{scrhack} % fix warnings when using KOMA with listings package
\usepackage{xspace} % to get the spacing after macros right
\usepackage{mparhack} % get marginpar right
\usepackage{fixltx2e} % fixes some LaTeX stuff
\usepackage[detect-all]{siunitx}%convenient units, DB
\sisetup{mode=text,range-phrase = {\text{~to~}}}%allows ranges to be used in math%DB
\usepackage{setspace}%allows double spacing for draft, DB
\usepackage{bm}%allows bold math, DB
%\usepackage{mcaption}%allows captions in margins, DB
\PassOptionsToPackage{printonlyused,smaller}{acronym}
	\usepackage{acronym} % nice macros for handling all acronyms in the thesis
%\renewcommand*{\acsfont}[1]{\textssc{#1}} % for MinionPro
\renewcommand{\bflabel}[1]{{#1}\hfill} % fix the list of acronyms
\usepackage{xcolor}

\PassOptionsToPackage{russian,american}{babel}   % change this to your language(s)
% Spanish languages need extra options in order to work with this template
%\PassOptionsToPackage{spanish,es-lcroman}{babel}
 \usepackage{babel}
% ****************************************************************************************************
%
% Some new commands that I like, DB
%
\renewcommand{\sup}[1]{\ensuremath{^{\mathrm{#1}}}}%DB
\newcommand  {\sub}[1]{\ensuremath{_{\mathrm{#1}}}}%DB
\newcommand  {\e}[1]{\ensuremath{\operatorname{e}^{#1}}}%DB
\newcommand  {\EcrossB}{\ensuremath{\vec{E}\times\vec{B}\;}}%DB
\newcommand  {\JcrossB}{\ensuremath{\vec{J}\times\vec{B}\;}}%DB
\newcommand  {\gradB}{\ensuremath{\nabla B\;}}%DB
\newcommand  {\infinity}{\ensuremath{\infty}}%DB
\newcommand  {\differential}[2]{\ensuremath{{\operatorname{d}\!{#1}\over\operatorname{d}\!{#2}}}}
\newcolumntype{x}[1]{>{\centering\hspace{0pt}}m{#1}}%DB
\newcolumntype{y}[1]{>{\raggedright\hspace{0pt}}m{#1}}%DB
\DeclareSIUnit{\atmosphere}{atm}%DB
\DeclareSIUnit{\amp}{A}%DB
\DeclareSIUnit{\torr}{torr}%DB
\DeclareSIUnit{\gauge}{AWG}%DB
\newcommand  {\Kocan}{Ko\v{c}an}%DB
\newcommand  {\inch}{\ensuremath{^{''}}}%DB
\newcommand  {\foot}{\ensuremath{^{'}}}%DB
\mathchardef\mhyphen="2D%DB good hyphen in math
\newcommand{\IV}{\ensuremath{I\mhyphen V}}%DB
\newcommand{\IVs}{\ensuremath{I\mhyphen V}s}%DB
\newcommand{\liningnums}[1]{{\fontfamily{pplx}\selectfont #1}}%DB
\newcommand{\CMod}{\mbox{C-Mod}}

\definecolor{myred}{rgb}{0.988,0.25,0.25}
\definecolor{RoyalBlue}{rgb}{0.988,0.25,0.25}

\makeatletter%DB
	\providecommand*{\diff}%
		{\@ifnextchar^{\DIfF}{\DIfF^{}}}
	\def\DIfF^#1{%
		\mathop{\mathrm{\mathstrut d}}%
			\nolimits^{#1}\gobblespace}
	\def\gobblespace{%
		\futurelet\diffarg\opspace}
	\def\opspace{%
		\let\DiffSpace\!%
		\ifx\diffarg(%
			\let\DiffSpace\relax
		\else
			\ifx\diffarg[%
				\let\DiffSpace\relax
			\else
				\ifx\diffarg\{%
					\let\DiffSpace\relax
				\fi\fi\fi\DiffSpace}
\newcommand*{\deriv}[3][]{%
	\frac{\diff^{#1}#2}{\diff #3^{#1}}}
\providecommand*{\pderiv}[3][]{%
	\frac{\partial^{#1}#2}%
		{\partial #3^{#1}}}
\makeatother

\newcommand*\myhrulefill{%
   \leavevmode\leaders\hrule depth-2pt height 2.4pt\hfill\kern0pt}

\newcommand\niceending[1]{%
  \begin{center}%
    \LARGE \myhrulefill \hspace{0.2cm} #1 \hspace{0.2cm} \myhrulefill%
  \end{center}}

\newcommand*\nicesectionending{\hfill\textcolor{myred}{$\bullet$}}
\newcommand*\nicechapterending{\hfill\textcolor{myred}{$\star$}}


% ****************************************************************************************************
% 4. Setup floats: tables, (sub)figures, and captions
% ****************************************************************************************************
\usepackage{tabularx} % better tables
	\setlength{\extrarowheight}{3pt} % increase table row height
\newcommand{\tableheadline}[1]{\multicolumn{1}{c}{\spacedlowsmallcaps{#1}}}
\newcommand{\myfloatalign}{\centering} % to be used with each float for alignment
%\usepackage{caption}
%\captionsetup{format=hang,font=small}
\usepackage[normal,bf,font=footnotesize,justification=raggedright]{caption}%DB
\usepackage{subfig}
% ****************************************************************************************************


% ****************************************************************************************************
% 5. Setup code listings
% ****************************************************************************************************
\usepackage{listings}
%\lstset{emph={trueIndex,root},emphstyle=\color{BlueViolet}}%\underbar} % for special keywords
\lstset{language=[LaTeX]Tex,%C++,
    keywordstyle=\color{RoyalBlue},%\bfseries,
    basicstyle=\small\ttfamily,
    %identifierstyle=\color{NavyBlue},
    commentstyle=\color{Green}\ttfamily,
    stringstyle=\rmfamily,
    numbers=none,%left,%
    numberstyle=\scriptsize,%\tiny
    stepnumber=5,
    numbersep=8pt,
    showstringspaces=false,
    breaklines=true,
    frameround=ftff,
    frame=single,
    belowcaptionskip=.75\baselineskip
    %frame=L
}
% ****************************************************************************************************    		 


% ****************************************************************************************************
% 6. PDFLaTeX, hyperreferences and citation backreferences
% ****************************************************************************************************
% ********************************************************************
% Using PDFLaTeX
% ********************************************************************
\PassOptionsToPackage{pdftex,hyperfootnotes=false,pdfpagelabels}{hyperref}
	\usepackage{hyperref}  % backref linktocpage pagebackref
\pdfcompresslevel=9
\pdfadjustspacing=1
\PassOptionsToPackage{pdftex}{graphicx}
	\usepackage{graphicx}

% ********************************************************************
% Setup the style of the backrefs from the bibliography
% (translate the options to any language you use)
% ********************************************************************
\newcommand{\backrefnotcitedstring}{\relax}%(Not cited.)
\newcommand{\backrefcitedsinglestring}[1]{(Cited on page~#1.)}
\newcommand{\backrefcitedmultistring}[1]{(Cited on pages~#1.)}
\ifthenelse{\boolean{enable-backrefs}}%
{%
		\PassOptionsToPackage{hyperpageref}{backref}
		\usepackage{backref} % to be loaded after hyperref package
		   \renewcommand{\backreftwosep}{ and~} % separate 2 pages
		   \renewcommand{\backreflastsep}{, and~} % separate last of longer list
		   \renewcommand*{\backref}[1]{}  % disable standard
		   \renewcommand*{\backrefalt}[4]{% detailed backref
		      \ifcase #1 %
		         \backrefnotcitedstring%
		      \or%
		         \backrefcitedsinglestring{#2}%
		      \else%
		         \backrefcitedmultistring{#2}%
		      \fi}%
}{\relax}

% ********************************************************************
% Hyperreferences
% ********************************************************************
\hypersetup{%
    %draft,	% = no hyperlinking at all (useful in b/w printouts)
    colorlinks=true, linktocpage=true, pdfstartpage=1, pdfstartview=FitV,%
    % uncomment the following line if you want to have black links (e.g., for printing)
    %colorlinks=false, linktocpage=false, pdfborder={0 0 0}, pdfstartpage=3, pdfstartview=FitV,%
    breaklinks=true, pdfpagemode=UseNone, pageanchor=true, pdfpagemode=UseOutlines,%
    plainpages=false, bookmarksnumbered, bookmarksopen=false, bookmarksopenlevel=1,%
    hypertexnames=true, pdfhighlight=/O,%nesting=true,%frenchlinks,%
    urlcolor=webbrown, linkcolor=RoyalBlue, citecolor=webgreen, %pagecolor=RoyalBlue,%
    %urlcolor=Black, linkcolor=Black, citecolor=Black, %pagecolor=Black,%
    pdftitle={\myTitle},%
    pdfauthor={\textcopyright\ \myName, \myUni, \myFaculty},%
    pdfsubject={},%
    pdfkeywords={},%
    pdfcreator={pdfLaTeX},%
    pdfproducer={LaTeX with hyperref and classicthesis}%
}

% ********************************************************************
% Setup autoreferences
% ********************************************************************
% There are some issues regarding autorefnames
% http://www.ureader.de/msg/136221647.aspx
% http://www.tex.ac.uk/cgi-bin/texfaq2html?label=latexwords
% you have to redefine the makros for the
% language you use, e.g., american, ngerman
% (as chosen when loading babel/AtBeginDocument)
% ********************************************************************
\makeatletter
\@ifpackageloaded{babel}%
    {%
       \addto\extrasamerican{%
					\renewcommand*{\figureautorefname}{Figure}%
					\renewcommand*{\tableautorefname}{Table}%
					\renewcommand*{\partautorefname}{Part}%
					\renewcommand*{\chapterautorefname}{Chapter}%
					\renewcommand*{\sectionautorefname}{Section}%
					\renewcommand*{\subsectionautorefname}{Section}%
					\renewcommand*{\subsubsectionautorefname}{Section}% 	
				}%
%        \addto\extrasngerman{%
% 					\renewcommand*{\paragraphautorefname}{Absatz}%
% 					\renewcommand*{\subparagraphautorefname}{Unterabsatz}%
% 					\renewcommand*{\footnoteautorefname}{Fu\"snote}%
% 					\renewcommand*{\FancyVerbLineautorefname}{Zeile}%
% 					\renewcommand*{\theoremautorefname}{Theorem}%
% 					\renewcommand*{\appendixautorefname}{Anhang}%
% 					\renewcommand*{\equationautorefname}{Gleichung}%
% 					\renewcommand*{\itemautorefname}{Punkt}%
% 				}%	
			% Fix to getting autorefs for subfigures right (thanks to Belinda Vogt for changing the definition)
			\providecommand{\subfigureautorefname}{\figureautorefname}%  			
    }{\relax}
\makeatother

% ****************************************************************************************************
% 7. Last calls before the bar closes
% ****************************************************************************************************
% ********************************************************************
% Development Stuff
% ********************************************************************
\listfiles
%\PassOptionsToPackage{l2tabu,orthodox,abort}{nag}
%	\usepackage{nag}
%\PassOptionsToPackage{warning, all}{onlyamsmath}
%	\usepackage{onlyamsmath}

% ********************************************************************
% Last, but not least...
% ********************************************************************
\usepackage{classicthesis}
% ****************************************************************************************************
\usepackage[capbesideposition={top}]{floatrow}   %allows adjustment of captions to fit figures %DB
\usepackage{wrapfig}    %allows figures to start in marge in protrude into text
\usepackage{calc}

\usepackage{changepage}
\newcommand{\pushtooutside}{%Checks if on odd or even page, pushes to align with outside margin of page
\strictpagecheck

\checkoddpage
\ifoddpage
  \hskip+2.0\marginspace
\else
  \hskip-2.0\marginspace
\fi
}

\newlength{\marginspace}
\setlength{\marginspace}{\marginparwidth+\marginparsep}
\setlength{\wrapoverhang}{\marginspace+\columnsep}

\makeatletter
\newcommand\margincaption[1]{%
  \par
  \setbox\@tempboxa=\hbox{\parbox{\marginparwidth}{\caption{#1}}}
  \@tempdima=\dimexpr\ht\@tempboxa+\dp\@tempboxa\relax
  \par\null\hfill\usebox\@tempboxa\par
  \vspace*{\dimexpr-\@tempdima-\baselineskip\relax}
}
\makeatother

\makeatletter
\newcommand*\smashcaption{%
        \def\FR@makecaption##1##2{%
                \vbox to\z@{%
                        \vss
                        \captionfont
                        {\captionlabelfont##1}\caption@lsep##2%
                        \par
                }%
        }%
        \caption
}
\makeatother

\setcounter{secnumdepth}{1}%only number sections %DB
\setcounter{tocdepth}{3}%include all levels in toc %DB

\usepackage[margincaption,outercaption,ragged,wide]{sidecap}

% rigth placed wrapfigure with right side caption
% wrapfigureScaption & wrapfigureRotScaption version 0.001
\newsavebox{\wImg}
\newsavebox{\wCap}
\newlength{\wImgH}
\newlength{\wImgW}
\newlength{\wCapH}
\newlength{\wCapW}
\newlength{\wImgTW}

\newcommand{\wrapfigureScaption}[3]{%
  \savebox{\wImg}{#1}
  \setlength{\wImgW}{\wd\wImg}
  \setlength{\wImgTW}{\wImgW}
  \setlength{\wCapW}{#3}
  \addtolength{\wImgTW}{\wCapW}
  \addtolength{\wImgTW}{10pt}
  \begin{wrapfigure}{r}{\wImgTW}
    \begin{minipage}[c]{\wImgW}
        \usebox{\wImg}
    \end{minipage}
    \hfill
    \begin{minipage}[c]{\wCapW}
        \caption{#2}
    \end{minipage}
  \end{wrapfigure}
}

%
% All the packages that I like, DB
%
\usepackage{amssymb}
\usepackage[page]{appendix}
\usepackage[loose]{units}
\usepackage[textwidth=2.5cm,textsize=small]{todonotes}
\usepackage{booktabs}
\usepackage{longtable}
\usepackage{pdflscape}
\usepackage{color}
\usepackage{chapterbib}
\usepackage{doi}


% Here's Gildea's Boilerplate Stuff.
% Copyright (c) 1987 by Stephen Gildea
% Permission to copy all or part of this work is granted, provided
% that the copies are not made or distributed for resale, and that
% the copyright notice and this notice are retained.

\makeatletter

\def\choosecase#1{#1}

%% Define all the pieces that go on the title page and the abstract.

% \title and \author already exist

\def\prevdegrees#1{\gdef\@prevdegrees{#1}}
\def\@prevdegrees{}

\def\department#1{\gdef\@department{#1}}

% If you are getting two degrees, use \and between the names.
\def\degree#1{\setbox0\hbox{#1}	 %for side effect of setting \@degreeword
  \gdef\@degree{#1}}

% \and is used inside the \degree argument to separate two degrees
\def\and{\gdef\@degreeword{degrees} \par and \par}
\def\@degreeword{degree}

% The copyright notice stuff is a tremendous mess.
%
% \@copyrightnotice is used by \maketitle to actually put text on the
% page; it defaults to ``Copyright MIT 19xx.  All rights reserved.''
% \copyrightnoticetext takes an argument and defined \@copyrightnotice
% to that argument.  \copyrightnotice takes an argument, and calls
% \copyrightnoticetext with that argument, preceeded by a copyright
% symbol and followed by ``All rights reserved.'' and the standard
% permission notice.
%
% If you use the 'vi' option, \copyrightnoticetext is used to set the
% copyright to ``(C) Your Name, Current Year in Roman Numerals.''
% followed by the permission notice.

% If there is no \copyrightnotice command, it is asssumed that MIT
% holds the copyright.  This commands adds the copyright symbol to the
% beginning, and puts the standard permission notice below.
%% ``All rights reserved'' added.  Krishna Sethuraman (1990)
\def\copyrightnotice#1{\copyrightnoticetext{\copyright\ #1.  All rights
reserved.\par\permission}}

% Occacionally you will need to exactly specify the text of the
% copyright notice.  The \copyrightnoticetext command is then useful.
\long\def\copyrightnoticetext#1{\gdef\@copyrightnotice{#1}}
\def\@copyrightnotice{\copyright\ \Mit\ \@degreeyear.  All rights reserved.}

%% `vi' documentclass option: Specifying this option automatically
%% copyrights the thesis to the author and gives MIT permission to copy and
%% distribute the document.  If you want, you can still specify
%% \copyrightnotice{stuff} to copyright to someone else, or
%% \copyrightnoticetext{stuff} to specify the exact text of the copyright
%% notice.
%%\ifodd\vithesis \copyrightnoticetext{\copyright\ \@author,
%%\uppercase\expandafter{\romannumeral\@degreeyear}.  All rights reserved.\par\permission}
%% or just
%%\@degreeyear}}
%%\typeout{Copyright given to author,
%%	permission to copy/distribute given to MIT.}
%%\else \typeout{Thesis document copyright MIT unless otherwise (manually) specified}
%%\fi

\def\thesisdate#1{\gdef\@thesisdate{#1}}

% typically just a month and year
\def\degreemonth#1{\gdef\@degreemonth{#1}}
\def\degreeyear#1{\gdef\@degreeyear{#1}}

% Usage: \supervisor{name}{title}
%        \chairman{name}{title}

% since there can be more than one supervisor,
% we build the appropriate boxes for the titlepage and
% the abstractpage as the user makes multiple calls
% to \supervisor
%\newbox\@titlesupervisor 	\newbox\@abstractsupervisor
%
%\def\supervisor#1#2{\setbox\@titlesupervisor\vbox
%  {\unvbox\@titlesupervisor \vskip 10pt% plus 1fil minus 1fil
%  \def\baselinestretch{1}\large
%  \signature{Certified by}{#1 \\ #2 \\ Thesis Supervisor}}
%  \setbox\@abstractsupervisor\vbox{\unvbox\@abstractsupervisor
%  \vskip\baselineskip \def\baselinestretch{1}\@normalsize
%  \par\noindent Thesis Supervisor: #1 \\ Title: #2}}

% supervisor
\def\supervisor#1#2{\gdef\@supervisorname{#1}\gdef\@supervisortitle{#2}}

% reader
\def\reader#1#2{\gdef\@readername{#1}\gdef\@readertitle{#2}}

% department chairman, not thesis committee chairman
\def\chairman#1#2{\gdef\@chairmanname{#1}\gdef\@chairmantitle{#2}}

%% `upcase' documentclass option: \choosecase is defined either as a dummy or
%% a macro to change the (expanded) argument to uppercase.
\def\maketitle{\begin{titlepage}
\large
{\def\baselinestretch{1.2}\Large\bf \choosecase{\@title} \par}
by\par
{\Large  \choosecase{\@author}}
\par
\@prevdegrees
\par
\choosecase{Submitted to the} \choosecase{\@department} \\
\choosecase{in partial fulfillment of the requirements for the}
\choosecase{\@degreeword}
\choosecase{of}
\par
\choosecase{\@degree}
\par
at the
\par\MIT\par
\@degreemonth\ \@degreeyear
\par
\@copyrightnotice
\par
\vskip 3\baselineskip
\signature{Author}{\@department \\ \@thesisdate}
\par
\vfill
\signature{Certified by}{\@supervisorname \\ \@supervisortitle \\ Thesis Supervisor}
\par
\vfill
\signature{Certified by}{\@readername \\ \@readertitle \\ Thesis Reader}
\par
\vfill
\signature{Accepted by}{\@chairmanname \\ \@chairmantitle}
\vfill
\end{titlepage}}

% this environment should probably be called abstract,
% but we want people to also be able to get at the more
% basic abstract environment
\def\abstractpage{\cleardoublepage
\begin{center}{\large{\bf \@title} \\
by \\
\@author \\[\baselineskip]}
\par
\def\baselinestretch{1}\@normalsize
Submitted to the \@department \\
on \@thesisdate, in partial fulfillment of the \\
requirements for the \@degreeword\ of \\
\@degree
\end{center}
\par
\begin{abstract}}

%% Changed from \unvbox to \unvcopy for use with multiple copies of abstract
%% page.
%% Krishna Sethuraman (1990)
\def\endabstractpage{\end{abstract}\noindent
    \vskip\baselineskip \def\baselinestretch{1}\@normalsize
    \par\noindent Thesis Supervisor: \@supervisorname \\ \@supervisortitle
    \vskip\baselineskip \def\baselinestretch{1}\@normalsize
    \par\noindent Thesis Reader: \@readername \\ \@readertitle
    \newpage
    }

%% This counter is used to save the page number for the second copy of
%% the abstract.
\newcounter{savepage}

% You can use the titlepage environment to do it all yourself if you
% don't want to use \maketitle.  If the titlepage environment, the
% paragraph skip is infinitely stretchable, so if you leave a blank line
% between lines that you want space between, the space will stretch so
% that the title page fills up the entire page.
\def\titlepage{\cleardoublepage\centering
  \thispagestyle{empty}
  \parindent 0pt \parskip 10pt plus 1fil minus 1fil
  \def\baselinestretch{1}\@normalsize\vbox to \vsize\bgroup\vbox to 9in\bgroup}
% The \kern0pt pushes any depth into the height.  Thanks to Richard Stone.
\def\endtitlepage{\par\kern 0pt\egroup\vss\egroup\newpage}

\def\MIT{MASSACHUSETTS INSTITUTE OF TECHNOLOGY}
\def\Mit{Massachusetts Institute of Technology}

\def\permission{\par\noindent{\centering
   The author hereby grants to MIT permission to reproduce and
   distribute publicly paper and electronic copies of this thesis
   document in whole or in part.}\par}

\def\signature#1#2{\par\noindent#1\dotfill\null\\*
  {\raggedleft #2\par}}

\def\abstract{\subsection*{Abstract}\small\def\baselinestretch{1}\@normalsize}
\def\endabstract{\par}

\makeatother

% ****************************************************************************************************
% 8. Further adjustments (experimental)
% ****************************************************************************************************
% ********************************************************************
% Changing the text area
% ********************************************************************
%\linespread{1.05} % a bit more for Palatino
%\areaset[current]{312pt}{761pt} % 686 (factor 2.2) + 33 head + 42 head \the\footskip
%\setlength{\marginparwidth}{7em}%
%\setlength{\marginparsep}{2em}%

% ********************************************************************
% Using different fonts
% ********************************************************************
%\usepackage[oldstylenums]{kpfonts} % oldstyle notextcomp
%\usepackage[osf]{libertine}
%\usepackage{hfoldsty} % Computer Modern with osf
%\usepackage[light,condensed,math]{iwona}
%\renewcommand{\sfdefault}{iwona}
%\usepackage{lmodern} % <-- no osf support :-(
%\usepackage[urw-garamond]{mathdesign} <-- no osf support :-(
% ****************************************************************************************************


% ****************************************************************************************************
% If you like the classicthesis, then I would appreciate a postcard.
% My address can be found in the file ClassicThesis.pdf. A collection
% of the postcards I received so far is available online at
% http://postcards.miede.de
% ****************************************************************************************************

% ****************************************************************************************************
% 1. Configure classicthesis for your needs here, e.g., remove "drafting" below
% in order to deactivate the time-stamp on the pages
% ****************************************************************************************************
%\PassOptionsToPackage{eulerchapternumbers,%drafting,%
%				 pdfspacing,%floatperchapter,%linedheaders,%
%				 subfig,beramono,eulermath}{classicthesis}		

\PassOptionsToPackage{eulerchapternumbers,drafting,%
				 pdfspacing,%floatperchapter,%linedheaders,%
				 subfig,eulermath}{classicthesis}									
% ********************************************************************
% Available options for classicthesis.sty
% (see ClassicThesis.pdf for more information):
% drafting
% parts nochapters linedheaders
% eulerchapternumbers beramono eulermath pdfspacing minionprospacing
% tocaligned dottedtoc manychapters
% listings floatperchapter subfig
% ********************************************************************

% ********************************************************************
% Triggers for this config
% ********************************************************************
\usepackage{ifthen}
\newboolean{enable-backrefs} % enable backrefs in the bibliography
\setboolean{enable-backrefs}{false} % true false
% ****************************************************************************************************


\usepackage{geometry}%[showframe] allows differnt margins on title page, DB

% ****************************************************************************************************
% 2. Personal data and user ad-hoc commands
% ****************************************************************************************************
\newcommand{\myTitle}{A Classic Thesis Style\xspace}
\newcommand{\mySubtitle}{An Homage to The Elements of Typographic Style\xspace}
\newcommand{\myDegree}{Doktor-Ingenieur (Dr.-Ing.)\xspace}
\newcommand{\myName}{Andr\'e Miede\xspace}
\newcommand{\myProf}{Put name here\xspace}
\newcommand{\myOtherProf}{Put name here\xspace}
\newcommand{\mySupervisor}{Put name here\xspace}
\newcommand{\myFaculty}{Put data here\xspace}
\newcommand{\myDepartment}{Put data here\xspace}
\newcommand{\myUni}{Put data here\xspace}
\newcommand{\myLocation}{Darmstadt\xspace}
\newcommand{\myTime}{August 2012\xspace}
\newcommand{\myVersion}{version 4.1\xspace}

% ********************************************************************
% Setup, finetuning, and useful commands
% ********************************************************************
\newcounter{dummy} % necessary for correct hyperlinks (to index, bib, etc.)
\newlength{\abcd} % for ab..z string length calculation
\providecommand{\mLyX}{L\kern-.1667em\lower.25em\hbox{Y}\kern-.125emX\@}
\newcommand{\ie}{i.\,e.,\;}
\newcommand{\Ie}{I.\,e.,\;}
\newcommand{\eg}{e.\,g.,\;}
\newcommand{\Eg}{E.\,g.,\;}
% ****************************************************************************************************


% ****************************************************************************************************
% 3. Loading some handy packages
% ****************************************************************************************************
% ********************************************************************
% Packages with options that might require adjustments
% ********************************************************************
\PassOptionsToPackage{latin9}{inputenc}	% latin9 (ISO-8859-9) = latin1+"Euro sign"
 \usepackage{inputenc}				

\PassOptionsToPackage{ngerman,american}{babel}   % change this to your language(s)
% Spanish languages need extra options in order to work with this template
%\PassOptionsToPackage{spanish,es-lcroman}{babel}
 \usepackage{babel}					

\PassOptionsToPackage{square,numbers,sort}{natbib}
 \usepackage{natbib}				

\PassOptionsToPackage{fleqn}{amsmath}		% math environments and more by the AMS
 \usepackage{amsmath}

% ********************************************************************
% General useful packages
% ********************************************************************
\PassOptionsToPackage{T1}{fontenc} % T2A for cyrillics
	\usepackage{fontenc}
\usepackage{textcomp} % fix warning with missing font shapes
\usepackage{scrhack} % fix warnings when using KOMA with listings package
\usepackage{xspace} % to get the spacing after macros right
\usepackage{mparhack} % get marginpar right
\usepackage{fixltx2e} % fixes some LaTeX stuff
\usepackage[detect-all]{siunitx}%convenient units, DB
\sisetup{mode=text,range-phrase = {\text{~to~}}}%allows ranges to be used in math%DB
\usepackage{setspace}%allows double spacing for draft, DB
\usepackage{bm}%allows bold math, DB
%\usepackage{mcaption}%allows captions in margins, DB
\PassOptionsToPackage{printonlyused,smaller}{acronym}
	\usepackage{acronym} % nice macros for handling all acronyms in the thesis
%\renewcommand*{\acsfont}[1]{\textssc{#1}} % for MinionPro
\renewcommand{\bflabel}[1]{{#1}\hfill} % fix the list of acronyms
\usepackage{xcolor}
% ****************************************************************************************************
%
% Some new commands that I like, DB
%
\renewcommand{\sup}[1]{\ensuremath{^{\mathrm{#1}}}}%DB
\newcommand  {\sub}[1]{\ensuremath{_{\mathrm{#1}}}}%DB
\newcommand  {\e}[1]{\ensuremath{\operatorname{e}^{#1}}}%DB
\newcommand  {\EcrossB}{\ensuremath{\vec{E}\times\vec{B}\;}}%DB
\newcommand  {\JcrossB}{\ensuremath{\vec{J}\times\vec{B}\;}}%DB
\newcommand  {\gradB}{\ensuremath{\nabla B\;}}%DB
\newcommand  {\infinity}{\ensuremath{\infty}}%DB
\newcommand  {\differential}[2]{\ensuremath{{\operatorname{d}\!{#1}\over\operatorname{d}\!{#2}}}}
\newcolumntype{x}[1]{>{\centering\hspace{0pt}}m{#1}}%DB
\newcolumntype{y}[1]{>{\raggedright\hspace{0pt}}m{#1}}%DB
\DeclareSIUnit{\atmosphere}{atm}%DB
\DeclareSIUnit{\amp}{A}%DB
\DeclareSIUnit{\torr}{torr}%DB
\DeclareSIUnit{\gauge}{AWG}%DB
\newcommand  {\Kocan}{Ko\v{c}an}%DB
\newcommand  {\inch}{\ensuremath{^{''}}}%DB
\newcommand  {\foot}{\ensuremath{^{'}}}%DB
\mathchardef\mhyphen="2D%DB good hyphen in math
\newcommand{\IV}{\ensuremath{I\mhyphen V}}%DB
\newcommand{\IVs}{\ensuremath{I\mhyphen V}s}%DB
\newcommand{\liningnums}[1]{{\fontfamily{pplx}\selectfont #1}}%DB
\newcommand{\CMod}{\mbox{C-Mod}}

\definecolor{myred}{rgb}{0.988,0.25,0.25}
\definecolor{RoyalBlue}{rgb}{0.988,0.25,0.25}

\makeatletter%DB
	\providecommand*{\diff}%
		{\@ifnextchar^{\DIfF}{\DIfF^{}}}
	\def\DIfF^#1{%
		\mathop{\mathrm{\mathstrut d}}%
			\nolimits^{#1}\gobblespace}
	\def\gobblespace{%
		\futurelet\diffarg\opspace}
	\def\opspace{%
		\let\DiffSpace\!%
		\ifx\diffarg(%
			\let\DiffSpace\relax
		\else
			\ifx\diffarg[%
				\let\DiffSpace\relax
			\else
				\ifx\diffarg\{%
					\let\DiffSpace\relax
				\fi\fi\fi\DiffSpace}
\newcommand*{\deriv}[3][]{%
	\frac{\diff^{#1}#2}{\diff #3^{#1}}}
\providecommand*{\pderiv}[3][]{%
	\frac{\partial^{#1}#2}%
		{\partial #3^{#1}}}
\makeatother

\newcommand*\myhrulefill{%
   \leavevmode\leaders\hrule depth-2pt height 2.4pt\hfill\kern0pt}

\newcommand\niceending[1]{%
  \begin{center}%
    \LARGE \myhrulefill \hspace{0.2cm} #1 \hspace{0.2cm} \myhrulefill%
  \end{center}}

\newcommand*\nicesectionending{\hfill\textcolor{myred}{$\bullet$}}
\newcommand*\nicechapterending{\hfill\textcolor{myred}{$\star$}}


% ****************************************************************************************************
% 4. Setup floats: tables, (sub)figures, and captions
% ****************************************************************************************************
\usepackage{tabularx} % better tables
	\setlength{\extrarowheight}{3pt} % increase table row height
\newcommand{\tableheadline}[1]{\multicolumn{1}{c}{\spacedlowsmallcaps{#1}}}
\newcommand{\myfloatalign}{\centering} % to be used with each float for alignment
%\usepackage{caption}
%\captionsetup{format=hang,font=small}
\usepackage[normal,bf,font=footnotesize,justification=raggedright]{caption}%DB
\usepackage{subfig}
% ****************************************************************************************************


% ****************************************************************************************************
% 5. Setup code listings
% ****************************************************************************************************
\usepackage{listings}
%\lstset{emph={trueIndex,root},emphstyle=\color{BlueViolet}}%\underbar} % for special keywords
\lstset{language=[LaTeX]Tex,%C++,
    keywordstyle=\color{RoyalBlue},%\bfseries,
    basicstyle=\small\ttfamily,
    %identifierstyle=\color{NavyBlue},
    commentstyle=\color{Green}\ttfamily,
    stringstyle=\rmfamily,
    numbers=none,%left,%
    numberstyle=\scriptsize,%\tiny
    stepnumber=5,
    numbersep=8pt,
    showstringspaces=false,
    breaklines=true,
    frameround=ftff,
    frame=single,
    belowcaptionskip=.75\baselineskip
    %frame=L
}
% ****************************************************************************************************    		 


% ****************************************************************************************************
% 6. PDFLaTeX, hyperreferences and citation backreferences
% ****************************************************************************************************
% ********************************************************************
% Using PDFLaTeX
% ********************************************************************
\PassOptionsToPackage{pdftex,hyperfootnotes=false,pdfpagelabels}{hyperref}
	\usepackage{hyperref}  % backref linktocpage pagebackref
\pdfcompresslevel=9
\pdfadjustspacing=1
\PassOptionsToPackage{pdftex}{graphicx}
	\usepackage{graphicx}

% ********************************************************************
% Setup the style of the backrefs from the bibliography
% (translate the options to any language you use)
% ********************************************************************
\newcommand{\backrefnotcitedstring}{\relax}%(Not cited.)
\newcommand{\backrefcitedsinglestring}[1]{(Cited on page~#1.)}
\newcommand{\backrefcitedmultistring}[1]{(Cited on pages~#1.)}
\ifthenelse{\boolean{enable-backrefs}}%
{%
		\PassOptionsToPackage{hyperpageref}{backref}
		\usepackage{backref} % to be loaded after hyperref package
		   \renewcommand{\backreftwosep}{ and~} % separate 2 pages
		   \renewcommand{\backreflastsep}{, and~} % separate last of longer list
		   \renewcommand*{\backref}[1]{}  % disable standard
		   \renewcommand*{\backrefalt}[4]{% detailed backref
		      \ifcase #1 %
		         \backrefnotcitedstring%
		      \or%
		         \backrefcitedsinglestring{#2}%
		      \else%
		         \backrefcitedmultistring{#2}%
		      \fi}%
}{\relax}

% ********************************************************************
% Hyperreferences
% ********************************************************************
\hypersetup{%
    %draft,	% = no hyperlinking at all (useful in b/w printouts)
    colorlinks=true, linktocpage=true, pdfstartpage=1, pdfstartview=FitV,%
    % uncomment the following line if you want to have black links (e.g., for printing)
    %colorlinks=false, linktocpage=false, pdfborder={0 0 0}, pdfstartpage=3, pdfstartview=FitV,%
    breaklinks=true, pdfpagemode=UseNone, pageanchor=true, pdfpagemode=UseOutlines,%
    plainpages=false, bookmarksnumbered, bookmarksopen=false, bookmarksopenlevel=1,%
    hypertexnames=true, pdfhighlight=/O,%nesting=true,%frenchlinks,%
    urlcolor=webbrown, linkcolor=RoyalBlue, citecolor=webgreen, %pagecolor=RoyalBlue,%
    %urlcolor=Black, linkcolor=Black, citecolor=Black, %pagecolor=Black,%
    pdftitle={\myTitle},%
    pdfauthor={\textcopyright\ \myName, \myUni, \myFaculty},%
    pdfsubject={},%
    pdfkeywords={},%
    pdfcreator={pdfLaTeX},%
    pdfproducer={LaTeX with hyperref and classicthesis}%
}

% ********************************************************************
% Setup autoreferences
% ********************************************************************
% There are some issues regarding autorefnames
% http://www.ureader.de/msg/136221647.aspx
% http://www.tex.ac.uk/cgi-bin/texfaq2html?label=latexwords
% you have to redefine the makros for the
% language you use, e.g., american, ngerman
% (as chosen when loading babel/AtBeginDocument)
% ********************************************************************
\makeatletter
\@ifpackageloaded{babel}%
    {%
       \addto\extrasamerican{%
					\renewcommand*{\figureautorefname}{Figure}%
					\renewcommand*{\tableautorefname}{Table}%
					\renewcommand*{\partautorefname}{Part}%
					\renewcommand*{\chapterautorefname}{Chapter}%
					\renewcommand*{\sectionautorefname}{Section}%
					\renewcommand*{\subsectionautorefname}{Section}%
					\renewcommand*{\subsubsectionautorefname}{Section}% 	
				}%
       \addto\extrasngerman{%
					\renewcommand*{\paragraphautorefname}{Absatz}%
					\renewcommand*{\subparagraphautorefname}{Unterabsatz}%
					\renewcommand*{\footnoteautorefname}{Fu\"snote}%
					\renewcommand*{\FancyVerbLineautorefname}{Zeile}%
					\renewcommand*{\theoremautorefname}{Theorem}%
					\renewcommand*{\appendixautorefname}{Anhang}%
					\renewcommand*{\equationautorefname}{Gleichung}%
					\renewcommand*{\itemautorefname}{Punkt}%
				}%	
			% Fix to getting autorefs for subfigures right (thanks to Belinda Vogt for changing the definition)
			\providecommand{\subfigureautorefname}{\figureautorefname}%  			
    }{\relax}
\makeatother

% ****************************************************************************************************
% 7. Last calls before the bar closes
% ****************************************************************************************************
% ********************************************************************
% Development Stuff
% ********************************************************************
\listfiles
%\PassOptionsToPackage{l2tabu,orthodox,abort}{nag}
%	\usepackage{nag}
%\PassOptionsToPackage{warning, all}{onlyamsmath}
%	\usepackage{onlyamsmath}

% ********************************************************************
% Last, but not least...
% ********************************************************************
\usepackage{classicthesis}
% ****************************************************************************************************
\usepackage[capbesideposition={top}]{floatrow}   %allows adjustment of captions to fit figures %DB
\usepackage{wrapfig}    %allows figures to start in marge in protrude into text
\usepackage{calc}

\usepackage{changepage}
\newcommand{\pushtooutside}{%Checks if on odd or even page, pushes to align with outside margin of page
\strictpagecheck

\checkoddpage
\ifoddpage
  \hskip+2.0\marginspace
\else
  \hskip-2.0\marginspace
\fi
}

\newlength{\marginspace}
\setlength{\marginspace}{\marginparwidth+\marginparsep}
\setlength{\wrapoverhang}{\marginspace+\columnsep}

\makeatletter
\newcommand\margincaption[1]{%
  \par
  \setbox\@tempboxa=\hbox{\parbox{\marginparwidth}{\caption{#1}}}
  \@tempdima=\dimexpr\ht\@tempboxa+\dp\@tempboxa\relax
  \par\null\hfill\usebox\@tempboxa\par
  \vspace*{\dimexpr-\@tempdima-\baselineskip\relax}
}
\makeatother

\makeatletter
\newcommand*\smashcaption{%
        \def\FR@makecaption##1##2{%
                \vbox to\z@{%
                        \vss
                        \captionfont
                        {\captionlabelfont##1}\caption@lsep##2%
                        \par
                }%
        }%
        \caption
}
\makeatother

\setcounter{secnumdepth}{1}%only number sections %DB
\setcounter{tocdepth}{3}%include all levels in toc %DB

\usepackage[margincaption,outercaption,ragged,wide]{sidecap}

% rigth placed wrapfigure with right side caption
% wrapfigureScaption & wrapfigureRotScaption version 0.001
\newsavebox{\wImg}
\newsavebox{\wCap}
\newlength{\wImgH}
\newlength{\wImgW}
\newlength{\wCapH}
\newlength{\wCapW}
\newlength{\wImgTW}

\newcommand{\wrapfigureScaption}[3]{%
  \savebox{\wImg}{#1}
  \setlength{\wImgW}{\wd\wImg}
  \setlength{\wImgTW}{\wImgW}
  \setlength{\wCapW}{#3}
  \addtolength{\wImgTW}{\wCapW}
  \addtolength{\wImgTW}{10pt}
  \begin{wrapfigure}{r}{\wImgTW}
    \begin{minipage}[c]{\wImgW}
        \usebox{\wImg}
    \end{minipage}
    \hfill
    \begin{minipage}[c]{\wCapW}
        \caption{#2}
    \end{minipage}
  \end{wrapfigure}
}

%
% All the packages that I like, DB
%
\usepackage{amssymb}
\usepackage[page]{appendix}
\usepackage[loose]{units}
\usepackage[textwidth=2.5cm,textsize=small]{todonotes}
\usepackage{booktabs}
\usepackage{longtable}
\usepackage{pdflscape}
\usepackage{color}
\usepackage{chapterbib}
\usepackage{doi}


% Here's Gildea's Boilerplate Stuff.
% Copyright (c) 1987 by Stephen Gildea
% Permission to copy all or part of this work is granted, provided
% that the copies are not made or distributed for resale, and that
% the copyright notice and this notice are retained.

\makeatletter

\def\choosecase#1{#1}

%% Define all the pieces that go on the title page and the abstract.

% \title and \author already exist

\def\prevdegrees#1{\gdef\@prevdegrees{#1}}
\def\@prevdegrees{}

\def\department#1{\gdef\@department{#1}}

% If you are getting two degrees, use \and between the names.
\def\degree#1{\setbox0\hbox{#1}	 %for side effect of setting \@degreeword
  \gdef\@degree{#1}}

% \and is used inside the \degree argument to separate two degrees
\def\and{\gdef\@degreeword{degrees} \par and \par}
\def\@degreeword{degree}

% The copyright notice stuff is a tremendous mess.
%
% \@copyrightnotice is used by \maketitle to actually put text on the
% page; it defaults to ``Copyright MIT 19xx.  All rights reserved.''
% \copyrightnoticetext takes an argument and defined \@copyrightnotice
% to that argument.  \copyrightnotice takes an argument, and calls
% \copyrightnoticetext with that argument, preceeded by a copyright
% symbol and followed by ``All rights reserved.'' and the standard
% permission notice.
%
% If you use the 'vi' option, \copyrightnoticetext is used to set the
% copyright to ``(C) Your Name, Current Year in Roman Numerals.''
% followed by the permission notice.

% If there is no \copyrightnotice command, it is asssumed that MIT
% holds the copyright.  This commands adds the copyright symbol to the
% beginning, and puts the standard permission notice below.
%% ``All rights reserved'' added.  Krishna Sethuraman (1990)
\def\copyrightnotice#1{\copyrightnoticetext{\copyright\ #1.  All rights
reserved.\par\permission}}

% Occacionally you will need to exactly specify the text of the
% copyright notice.  The \copyrightnoticetext command is then useful.
\long\def\copyrightnoticetext#1{\gdef\@copyrightnotice{#1}}
\def\@copyrightnotice{\copyright\ \Mit\ \@degreeyear.  All rights reserved.}

%% `vi' documentclass option: Specifying this option automatically
%% copyrights the thesis to the author and gives MIT permission to copy and
%% distribute the document.  If you want, you can still specify
%% \copyrightnotice{stuff} to copyright to someone else, or
%% \copyrightnoticetext{stuff} to specify the exact text of the copyright
%% notice.
%%\ifodd\vithesis \copyrightnoticetext{\copyright\ \@author,
%%\uppercase\expandafter{\romannumeral\@degreeyear}.  All rights reserved.\par\permission}
%% or just
%%\@degreeyear}}
%%\typeout{Copyright given to author,
%%	permission to copy/distribute given to MIT.}
%%\else \typeout{Thesis document copyright MIT unless otherwise (manually) specified}
%%\fi

\def\thesisdate#1{\gdef\@thesisdate{#1}}

% typically just a month and year
\def\degreemonth#1{\gdef\@degreemonth{#1}}
\def\degreeyear#1{\gdef\@degreeyear{#1}}

% Usage: \supervisor{name}{title}
%        \chairman{name}{title}

% since there can be more than one supervisor,
% we build the appropriate boxes for the titlepage and
% the abstractpage as the user makes multiple calls
% to \supervisor
%\newbox\@titlesupervisor 	\newbox\@abstractsupervisor
%
%\def\supervisor#1#2{\setbox\@titlesupervisor\vbox
%  {\unvbox\@titlesupervisor \vskip 10pt% plus 1fil minus 1fil
%  \def\baselinestretch{1}\large
%  \signature{Certified by}{#1 \\ #2 \\ Thesis Supervisor}}
%  \setbox\@abstractsupervisor\vbox{\unvbox\@abstractsupervisor
%  \vskip\baselineskip \def\baselinestretch{1}\@normalsize
%  \par\noindent Thesis Supervisor: #1 \\ Title: #2}}

% supervisor
\def\supervisor#1#2{\gdef\@supervisorname{#1}\gdef\@supervisortitle{#2}}

% reader
\def\reader#1#2{\gdef\@readername{#1}\gdef\@readertitle{#2}}

% department chairman, not thesis committee chairman
\def\chairman#1#2{\gdef\@chairmanname{#1}\gdef\@chairmantitle{#2}}

%% `upcase' documentclass option: \choosecase is defined either as a dummy or
%% a macro to change the (expanded) argument to uppercase.
\def\maketitle{\begin{titlepage}
\large
{\def\baselinestretch{1.2}\Large\bf \choosecase{\@title} \par}
by\par
{\Large  \choosecase{\@author}}
\par
\@prevdegrees
\par
\choosecase{Submitted to the} \choosecase{\@department} \\
\choosecase{in partial fulfillment of the requirements for the}
\choosecase{\@degreeword}
\choosecase{of}
\par
\choosecase{\@degree}
\par
at the
\par\MIT\par
\@degreemonth\ \@degreeyear
\par
\@copyrightnotice
\par
\vskip 3\baselineskip
\signature{Author}{\@department \\ \@thesisdate}
\par
\vfill
\signature{Certified by}{\@supervisorname \\ \@supervisortitle \\ Thesis Supervisor}
\par
\vfill
\signature{Certified by}{\@readername \\ \@readertitle \\ Thesis Reader}
\par
\vfill
\signature{Accepted by}{\@chairmanname \\ \@chairmantitle}
\vfill
\end{titlepage}}

% this environment should probably be called abstract,
% but we want people to also be able to get at the more
% basic abstract environment
\def\abstractpage{\cleardoublepage
\begin{center}{\large{\bf \@title} \\
by \\
\@author \\[\baselineskip]}
\par
\def\baselinestretch{1}\@normalsize
Submitted to the \@department \\
on \@thesisdate, in partial fulfillment of the \\
requirements for the \@degreeword\ of \\
\@degree
\end{center}
\par
\begin{abstract}}

%% Changed from \unvbox to \unvcopy for use with multiple copies of abstract
%% page.
%% Krishna Sethuraman (1990)
\def\endabstractpage{\end{abstract}\noindent
    \vskip\baselineskip \def\baselinestretch{1}\@normalsize
    \par\noindent Thesis Supervisor: \@supervisorname \\ \@supervisortitle
    \vskip\baselineskip \def\baselinestretch{1}\@normalsize
    \par\noindent Thesis Reader: \@readername \\ \@readertitle
    \newpage
    }

%% This counter is used to save the page number for the second copy of
%% the abstract.
\newcounter{savepage}

% You can use the titlepage environment to do it all yourself if you
% don't want to use \maketitle.  If the titlepage environment, the
% paragraph skip is infinitely stretchable, so if you leave a blank line
% between lines that you want space between, the space will stretch so
% that the title page fills up the entire page.
\def\titlepage{\cleardoublepage\centering
  \thispagestyle{empty}
  \parindent 0pt \parskip 10pt plus 1fil minus 1fil
  \def\baselinestretch{1}\@normalsize\vbox to \vsize\bgroup\vbox to 9in\bgroup}
% The \kern0pt pushes any depth into the height.  Thanks to Richard Stone.
\def\endtitlepage{\par\kern 0pt\egroup\vss\egroup\newpage}

\def\MIT{MASSACHUSETTS INSTITUTE OF TECHNOLOGY}
\def\Mit{Massachusetts Institute of Technology}

\def\permission{\par\noindent{\centering
   The author hereby grants to MIT permission to reproduce and
   distribute publicly paper and electronic copies of this thesis
   document in whole or in part.}\par}

\def\signature#1#2{\par\noindent#1\dotfill\null\\*
  {\raggedleft #2\par}}

\def\abstract{\subsection*{Abstract}\small\def\baselinestretch{1}\@normalsize}
\def\endabstract{\par}

\makeatother

% ****************************************************************************************************
% 8. Further adjustments (experimental)
% ****************************************************************************************************
% ********************************************************************
% Changing the text area
% ********************************************************************
%\linespread{1.05} % a bit more for Palatino
%\areaset[current]{312pt}{761pt} % 686 (factor 2.2) + 33 head + 42 head \the\footskip
%\setlength{\marginparwidth}{7em}%
%\setlength{\marginparsep}{2em}%

% ********************************************************************
% Using different fonts
% ********************************************************************
%\usepackage[oldstylenums]{kpfonts} % oldstyle notextcomp
%\usepackage[osf]{libertine}
%\usepackage{hfoldsty} % Computer Modern with osf
%\usepackage[light,condensed,math]{iwona}
%\renewcommand{\sfdefault}{iwona}
%\usepackage{lmodern} % <-- no osf support :-(
%\usepackage[urw-garamond]{mathdesign} <-- no osf support :-(
% ****************************************************************************************************

